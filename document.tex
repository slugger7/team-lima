\documentclass[a4paper,12pt]{report}
\addtolength{\oddsidemargin}{-1.cm}
\addtolength{\textwidth}{2cm}
\addtolength{\topmargin}{-2cm}
\addtolength{\textheight}{3.5cm}
\newcommand{\HRule}{\rule{\linewidth}{0.5mm}}
\makeindex

\usepackage{longtable}
\usepackage[pdftex]{graphicx}
\usepackage{makeidx}
\usepackage{hyperref}
\usepackage{verbatim}
\hypersetup{
    colorlinks=true,
    linkcolor=blue,
    filecolor=magenta,      
    urlcolor=cyan,
}


% define the title
\author{Team Lima}
\title{ Assignment 1}
\begin{document}
\setlength{\parskip}{6pt}

% generates the title
\begin{titlepage}

\begin{center}
% Upper part of the page       
\includegraphics[width=1\textwidth]{./up-logo.jpg}\\[0.4cm]    
\textsc{\LARGE Department of Computer Science}\\[1.5cm]
\textsc{\Large COS 301 - Mini Project}\\[0.5cm]
% Title
\HRule \\[0.4cm]
{ \huge \bfseries Assignment 1}\\[0.4cm]
\HRule \\[0.4cm]
% Author and supervisor
\begin{minipage}{0.4\textwidth}
\begin{flushleft} \large
\emph{Author:}\\
Tshepo {Malesela}
\end{flushleft}
\end{minipage}
\begin{minipage}{0.4\textwidth}
\begin{flushright} \large
\emph{Student number:} \\
u14211582
\end{flushright}
\end{minipage}
\begin{minipage}{0.4\textwidth}
\begin{flushleft} \large
\emph{} \\
Kevin David {Heritage}
\end{flushleft}
\end{minipage}
\begin{minipage}{0.4\textwidth}
\begin{flushright} \large
\emph{} \\
u13044924
\end{flushright}
\end{minipage}
\begin{minipage}{0.4\textwidth}
\begin{flushleft} \large
Unarine {Rambani}
\end{flushleft}
\end{minipage}
\begin{minipage}{0.4\textwidth}
\begin{flushright} \large
\emph{} \\
u14004489
\end{flushright}
\end{minipage}
\begin{minipage}{0.4\textwidth}
\begin{flushleft} \large
Vukile {Langa}
\end{flushleft}
\end{minipage}
\begin{minipage}{0.4\textwidth}
\begin{flushright} \large
\emph{} \\
u14035449 
\end{flushright}
\end{minipage}
\begin{minipage}{0.4\textwidth}
\begin{flushleft} \large
Wynand Hugo {Meiring}
\end{flushleft}
\end{minipage}
\begin{minipage}{0.4\textwidth}
\begin{flushright} \large
\emph{} \\
u13230795  
\end{flushright}
\end{minipage}
\begin{minipage}{0.4\textwidth}
\begin{flushleft} \large
Ntokozo
\end{flushleft}
\end{minipage}
\begin{minipage}{0.4\textwidth}
\begin{flushright} \large
\emph{} \\
u14414555
\end{flushright}
\end{minipage}
\begin{minipage}{0.4\textwidth}
\begin{flushleft} \large
Unknown
\end{flushleft}
\end{minipage}
\begin{minipage}{0.4\textwidth}
\begin{flushright} \large
\emph{} \\
Unknown
\end{flushright}
\end{minipage}
\vfill

{\large \today}
\end{center}
\end{titlepage}
\footnotesize
\normalsize

\renewcommand{\thesection}{\arabic{section}}
\newpage
\begin{center}
\textsc{\LARGE Requirements Specification}\\[1.5cm]
\textsc{\Large Team Lima github repository link}\\[0.5cm]
For further references see \href{https://https://github.com/slugger7/team-lima}{gitHub}.
\today
\end{center}


\newpage
\section{Introduction}
In this section we will explore an overview of everything that is included in the Software Requirements Specification (SRS) document.

\subsection{Purpose}
The purpose of this document is to give a detailed description of the features of the Mini-Project, to serve as a guide or reference for the software developers and primarily a proposal to the client approval.

\subsection{Scope}
The Mini-Project(Publication System) is a web and Android application which will assist research leaders as well as the heads of department to keep track of the progress of different research groups.
\newline Users(UP Staff) will act as the administrators for the different research groups and provide information regarding research topics, group members, progress, etc using the web-portal.
\newline The system will only keep track of metadata and not contain any of the actual publications. The information is maintained in a database that is located on a web-server. An internet connection will be required to fetch and display the information.

\subsection{Overview}

The remainder of this document includes four more sections and is organized as follows:
\newline Section 2 describes the Vision, which basically explains what the client wants to achieve with the project and what the average user would get out of the product.Section 3 continues to discuss the background, which includes the business/ research opportunity to simplify the administration and management of documents/publications in the world of research. This section also sheds light on the problems being faced that may have led to this project being started. Section 4 presents the Architecture requirements to the reader, including the integration and quality requirements as well as the architectural constraints. Section 5 describes the Functional Requirements and captures all the functionality which would be required by the users of the system. Section 6 brings up the Open Issues which entail anything that needs to be clarified regarding the requirements as well as any irregularities/inconsistencies found in the requirements.


\newpage
\section{Vision}

\newpage
\section{Background}

\newpage
\section{Architecture requirements}

\newpage
\section{Functional requirements and application design}
\subsection{Use case prioritization}
\subsubsection{Critical}
	\begin{itemize}
		\item The system needs to be able to handle 100 users concurrently and at the same time.
		\item Only University of Pretoria staff members may register to become users.
		\item Users are only allowed to see publications that they are involved in.
		\item There should be a a superuser or root account for the Head of Department, the superuser should be able to see all research papers that is on the system.
	\end{itemize}

\subsubsection{Important}
	\begin{itemize}
		\item Users(UP staff) can add and remove authors.
		\item Users(UP staff) decide when the deadline of a research paper is.
		\item Users(UP staff) can terminate a research paper and revive a terminated research paper.
		\item A user can edit the details about a research paper that they are involved in.
		\item User profiles should show the total accumulated units.
		\item The system has to log everything.
		\item Users must be able to enter historical data (previously completed research papers).
		\item Every research paper only has one primary author.
		\item Co-authors are added and sequenced.
	\end{itemize}

\subsubsection{Nice-To-Have}
	\begin{itemize}
		\item There should be a checkbox that defaults users as authors for research papers that they administrate, if they are just administrating they should explicitly uncheck that they are not an author.
		\item There should be an indication of progress on research papers (progress bar/percentage indication of work done)
		\item Users should be able to search for authors if they have been stored in the system.
	\end{itemize}

\newpage
\section{Open Issues}
\begin{itemize}
\item It was never specified whether researchers(non-users) would be able to see the information about the publications they were involved in
\item Still unclear whether researchers(both users and non-users) will be notified if they are removed from a research group
\item We are not sure if a user can have multiple sessions open concurrently - if a user is already logged in on the web application, will they be able to log in using their mobile device?
\item Will the progress be represented by a percentage / a progress bar?
\item Does the user set the order of the authors or is it automatically set when the user inputs these into a specific document

\item Is there an option that will be provided for how long before a deadline the user will want to receive a notification?
\item Won't privacy be an issue if researcher(non-user) information is put up and every user - even those not in the same research group - is able to view it?
\item Will there be different tabs for different types of publications a user is busy with? ie. screens they can switch between to check on articles, Case reports, terminated publications, etc
\item Will there be a way that research group members can access each other's information so that they can make contact?
\item What other applications or platforms must this system be able to integrate with?
\end{itemize}

\newpage
\section{References}


\end{document}

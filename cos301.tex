\documentclass[a4paper,12pt]{report}

\begin{document}
\chapter{Vision}
\section{Product Functions}

The main aim of this system is to assist researchers(authors) and head of department with storing metadata of research papers(not the paper itself). In the system the users will be able to see how many papers is that particular user working on,  identify which research papers  are due at what time(the system will keep a daily reminder of the due date) and who are the other author of the paper but will maintain privacy since a user won't be able to see other user's work(only going to see papers where the user co-author) unless it is the head of department or a supervisor.The system will keep a status of the paper(still working on it,submitted waiting for feedback,rejected, accepted,published or improving it after rejection).

The system should be able support 100 users concurrently without any difficulties.The head of department is allowed to see everything in the system meaning everyone's work. An author is not necessarily a user of the system meaning that a person can be added as one of the authors of a certain research paper, while they are not a user of the system, the primary author is responsible of adding and deleting authors for a paper and the primary should be a user of the system.

Every research paper on the system should have at least one user responsible for it. The system keeps the title of the paper, if a paper is accepted/published a link to where the actual paper is stored,if a paper is deleted in the system(by the primary author) a reason, should be provided for that termination, the type of a paper(journal, conference paper or book chapter) is also kept in the system, and the system should also keep report containing DoE Units, DoE Research Output Units,UPWeighted Research Outputs, Estimated DoE Outputs and Research Funding.

\section{User classes and Characteristics}
\subsection{Head of Department}
\begin{itemize}
\item
Must be a registered user of the user of the system.
\item
There is only one head of Department, since the system is for one department.
\item
 Able to see everyone's work on the system.
\item
May be an author.
\item
May add or delete papers.

\end{itemize}
\subsection{Primary Author/ Leading Researcher}
\begin{itemize}
\item 
Must be a registered user of the user of the system.
\item
Any number less than a hundred.
\item
Able to see only the list papers of where is authoring or co-authoring.
\item
Can add or delete papers.
\item
Can add or delete authors.
\end{itemize}

\subsection{Author}
\begin{itemize}
\item 
May or may not be a user of the system.
\item
If a user able to see only the list papers of where is co-authoring.
\item
Otherwise not able to see anything in the system.
\end{itemize}

\section{Operating Environment}
The system will be a standalone system, it does not load any external module or library function. For the user to login to the system need to have internet access.When a user modify something every other use who have have access to the piece of information need access the latest updated version of the information. The system must run on any operating system, and it will have make use of a virtual interface for the different operating system and hardware platforms.

\chapter{Background}
The system give  different authors/researchers an opporturnity to work on the same paper and be able to stay updated about the progress of the paper. The system also gives the head of department an opporturnity to better supervise the stuff member on the researches they are working on and it is important for the head of department to know everything about the researches in the department, since research is one of the crucial aspects of every department in the university. The system also give the head of department an opportunity to evalute the staff member using the report the system provides.

A correct usage of the system will yield good results in management of the department's researches and will also provide a better way of co-authoring among different authors/users because when there is a change in the status of the paper all author will know and every user will be aware of the deadline of their work.

The system will also improve the current way of storing research metadata, that the department is using which is  writing everything on a spreadSheet, the system will also add other things that are not in the spreadSheet like a full report of how the whole department performing with regards to research papers and provided a more informed way of co-author, since everyone work in the paper will know what is going on the paper.

\end{document}
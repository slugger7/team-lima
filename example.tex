\documentclass[a4paper,12pt]{report}
\addtolength{\oddsidemargin}{-1.cm}
\addtolength{\textwidth}{2cm}
\addtolength{\topmargin}{-2cm}
\addtolength{\textheight}{3.5cm}
\newcommand{\HRule}{\rule{\linewidth}{0.5mm}}
\makeindex
\setcounter{secnumdepth}{5}

\usepackage{longtable}
\usepackage[utf8]{inputenc}
\usepackage[T1]{fontenc}
\usepackage[pdftex]{graphicx}
\usepackage{makeidx}
\usepackage{hyperref}
\hypersetup{
    colorlinks=true,
    linkcolor=blue,
    filecolor=magenta,      
    urlcolor=cyan,
}


% define the title
\author{Group1_a}
\title{ Assignment 1 Report}
\begin{document}
\setlength{\parskip}{6pt}

% generates the title
\begin{titlepage}

\begin{center}
% Upper part of the page       
\includegraphics[width=1\textwidth]{./up-logo.jpg}\\[0.4cm]    
\textsc{\LARGE Department of Computer Science}\\[1.5cm]
\textsc{\Large COS 301 - Software Engineering}\\[0.5cm]
% Title
\HRule \\[0.4cm]
{ \huge \bfseries COS 301 - Mini Project}\\[0.4cm]
\HRule \\[0.4cm]
% Author and supervisor
\begin{minipage}{0.4\textwidth}
\begin{flushleft} \large
\emph{Author:}\\
Una{Rambani}
\end{flushleft}
\end{minipage}
\begin{minipage}{0.4\textwidth}
\begin{flushright} \large
\emph{Student number:} \\
u14004489
\end{flushright}
\end{minipage}
\begin{minipage}{0.4\textwidth}
\begin{flushleft} \large
kevin {Heritage}
\end{flushleft}
\end{minipage}
\begin{minipage}{0.4\textwidth}
\begin{flushright} \large
\emph{} \\
u13044924
\end{flushright}
\end{minipage}
\begin{minipage}{0.4\textwidth}
\begin{flushleft} \large
Tshepo {Malesela}
\end{flushleft}
\end{minipage}
\begin{minipage}{0.4\textwidth}
\begin{flushright} \large
\emph{} \\
u14211582
\end{flushright}
\end{minipage}
\begin{minipage}{0.4\textwidth}
\begin{flushleft} \large
Vukile {Langa}
\end{flushleft}
\end{minipage}
\begin{minipage}{0.4\textwidth}
\begin{flushright} \large
\emph{} \\
u14035449
\end{flushright}
\end{minipage}
\begin{minipage}{0.4\textwidth}
\begin{flushleft} \large
Wynand {Meirng}
\end{flushleft}
\end{minipage}
\begin{minipage}{0.4\textwidth}
\begin{flushright} \large
\emph{} \\
u13230795
\end{flushright}
\end{minipage}
\begin{minipage}{0.4\textwidth}
\begin{flushleft} \large
Nontokozo {Hlatshwayo}
\end{flushleft}
\end{minipage}
\begin{minipage}{0.4\textwidth}
\begin{flushright} \large
\emph{} \\
u14414555
\end{flushright}
\end{minipage}
\begin{minipage}{0.4\textwidth}
\begin{flushleft} \large
Tim {Kirker}
\end{flushleft}
\end{minipage}
\begin{minipage}{0.4\textwidth}
\begin{flushright} \large
\emph{} \\
u11152402
\end{flushright}
\end{minipage}
\begin{minipage}{0.4\textwidth}
\begin{flushleft} \large
Thabang {Letageng}
\end{flushleft}
\end{minipage}
\begin{minipage}{0.4\textwidth}
\begin{flushright} \large
\emph{} \\
u13057937
\end{flushright}
\end{minipage}
\begin{minipage}{0.4\textwidth}
\begin{flushleft} \large
Antonia {Michael}
\end{flushleft}
\end{minipage}
\begin{minipage}{0.4\textwidth}
\begin{flushright} \large
\emph{} \\
u13014171
\end{flushright}
\end{minipage}
\vfill
% Bottom of the page
{\large \today}
\end{center}
\end{titlepage}
\footnotesize
\documentclass[a4paper,12pt]{report}
\addtolength{\oddsidemargin}{-1.cm}
\addtolength{\textwidth}{2cm}
\addtolength{\topmargin}{-2cm}
\addtolength{\textheight}{3.5cm}
\newcommand{\HRule}{\rule{\linewidth}{0.5mm}}
\makeindex
\setcounter{secnumdepth}{5}

\usepackage{longtable}
\usepackage[utf8]{inputenc}
\usepackage[T1]{fontenc}
\usepackage[pdftex]{graphicx}
\usepackage{makeidx}
\usepackage{hyperref}
\hypersetup{
    colorlinks=true,
    linkcolor=blue,
    filecolor=magenta,      
    urlcolor=cyan,
}


% define the title
\author{Group1_a}
\title{ Assignment 1 Report}
\begin{document}
\setlength{\parskip}{6pt}

% generates the title
\begin{titlepage}

\begin{center}
% Upper part of the page       
\includegraphics[width=1\textwidth]{./up-logo.jpg}\\[0.4cm]    
\textsc{\LARGE Department of Computer Science}\\[1.5cm]
\textsc{\Large COS 301 - Software Engineering}\\[0.5cm]
% Title
\HRule \\[0.4cm]
{ \huge \bfseries COS 301 - Mini Project}\\[0.4cm]
\HRule \\[0.4cm]
% Author and supervisor
\begin{minipage}{0.4\textwidth}
\begin{flushleft} \large
\emph{Author:}\\
Una{Rambani}
\end{flushleft}
\end{minipage}
\begin{minipage}{0.4\textwidth}
\begin{flushright} \large
\emph{Student number:} \\
u14004489
\end{flushright}
\end{minipage}
\begin{minipage}{0.4\textwidth}
\begin{flushleft} \large
kevin {Heritage}
\end{flushleft}
\end{minipage}
\begin{minipage}{0.4\textwidth}
\begin{flushright} \large
\emph{} \\
u13044924
\end{flushright}
\end{minipage}
\begin{minipage}{0.4\textwidth}
\begin{flushleft} \large
Tshepo {Malesela}
\end{flushleft}
\end{minipage}
\begin{minipage}{0.4\textwidth}
\begin{flushright} \large
\emph{} \\
u14211582
\end{flushright}
\end{minipage}
\begin{minipage}{0.4\textwidth}
\begin{flushleft} \large
Vukile {Langa}
\end{flushleft}
\end{minipage}
\begin{minipage}{0.4\textwidth}
\begin{flushright} \large
\emph{} \\
u14035449
\end{flushright}
\end{minipage}
\begin{minipage}{0.4\textwidth}
\begin{flushleft} \large
Wynand {Meirng}
\end{flushleft}
\end{minipage}
\begin{minipage}{0.4\textwidth}
\begin{flushright} \large
\emph{} \\
u13230795
\end{flushright}
\end{minipage}
\begin{minipage}{0.4\textwidth}
\begin{flushleft} \large
Nontokozo {Hlatshwayo}
\end{flushleft}
\end{minipage}
\begin{minipage}{0.4\textwidth}
\begin{flushright} \large
\emph{} \\
u14414555
\end{flushright}
\end{minipage}
\begin{minipage}{0.4\textwidth}
\begin{flushleft} \large
Tim {Kirker}
\end{flushleft}
\end{minipage}
\begin{minipage}{0.4\textwidth}
\begin{flushright} \large
\emph{} \\
u11152402
\end{flushright}
\end{minipage}
\begin{minipage}{0.4\textwidth}
\begin{flushleft} \large
Thabang {Letageng}
\end{flushleft}
\end{minipage}
\begin{minipage}{0.4\textwidth}
\begin{flushright} \large
\emph{} \\
u13057937
\end{flushright}
\end{minipage}
\begin{minipage}{0.4\textwidth}
\begin{flushleft} \large
Antonia {Michael}
\end{flushleft}
\end{minipage}
\begin{minipage}{0.4\textwidth}
\begin{flushright} \large
\emph{} \\
u13014171
\end{flushright}
\end{minipage}
\vfill
% Bottom of the page
{\large \today}
\end{center}
\end{titlepage}
\footnotesize
\documentclass[a4paper,12pt]{report}
\addtolength{\oddsidemargin}{-1.cm}
\addtolength{\textwidth}{2cm}
\addtolength{\topmargin}{-2cm}
\addtolength{\textheight}{3.5cm}
\newcommand{\HRule}{\rule{\linewidth}{0.5mm}}
\makeindex
\setcounter{secnumdepth}{5}

\usepackage{longtable}
\usepackage[utf8]{inputenc}
\usepackage[T1]{fontenc}
\usepackage[pdftex]{graphicx}
\usepackage{makeidx}
\usepackage{hyperref}
\hypersetup{
    colorlinks=true,
    linkcolor=blue,
    filecolor=magenta,      
    urlcolor=cyan,
}


% define the title
\author{Group1_a}
\title{ Assignment 1 Report}
\begin{document}
\setlength{\parskip}{6pt}

% generates the title
\begin{titlepage}

\begin{center}
% Upper part of the page       
\includegraphics[width=1\textwidth]{./up-logo.jpg}\\[0.4cm]    
\textsc{\LARGE Department of Computer Science}\\[1.5cm]
\textsc{\Large COS 301 - Software Engineering}\\[0.5cm]
% Title
\HRule \\[0.4cm]
{ \huge \bfseries COS 301 - Mini Project}\\[0.4cm]
\HRule \\[0.4cm]
% Author and supervisor
\begin{minipage}{0.4\textwidth}
\begin{flushleft} \large
\emph{Author:}\\
Una{Rambani}
\end{flushleft}
\end{minipage}
\begin{minipage}{0.4\textwidth}
\begin{flushright} \large
\emph{Student number:} \\
u14004489
\end{flushright}
\end{minipage}
\begin{minipage}{0.4\textwidth}
\begin{flushleft} \large
kevin {Heritage}
\end{flushleft}
\end{minipage}
\begin{minipage}{0.4\textwidth}
\begin{flushright} \large
\emph{} \\
u13044924
\end{flushright}
\end{minipage}
\begin{minipage}{0.4\textwidth}
\begin{flushleft} \large
Tshepo {Malesela}
\end{flushleft}
\end{minipage}
\begin{minipage}{0.4\textwidth}
\begin{flushright} \large
\emph{} \\
u14211582
\end{flushright}
\end{minipage}
\begin{minipage}{0.4\textwidth}
\begin{flushleft} \large
Vukile {Langa}
\end{flushleft}
\end{minipage}
\begin{minipage}{0.4\textwidth}
\begin{flushright} \large
\emph{} \\
u14035449
\end{flushright}
\end{minipage}
\begin{minipage}{0.4\textwidth}
\begin{flushleft} \large
Wynand {Meirng}
\end{flushleft}
\end{minipage}
\begin{minipage}{0.4\textwidth}
\begin{flushright} \large
\emph{} \\
u13230795
\end{flushright}
\end{minipage}
\begin{minipage}{0.4\textwidth}
\begin{flushleft} \large
Nontokozo {Hlatshwayo}
\end{flushleft}
\end{minipage}
\begin{minipage}{0.4\textwidth}
\begin{flushright} \large
\emph{} \\
u14414555
\end{flushright}
\end{minipage}
\begin{minipage}{0.4\textwidth}
\begin{flushleft} \large
Tim {Kirker}
\end{flushleft}
\end{minipage}
\begin{minipage}{0.4\textwidth}
\begin{flushright} \large
\emph{} \\
u11152402
\end{flushright}
\end{minipage}
\begin{minipage}{0.4\textwidth}
\begin{flushleft} \large
Thabang {Letageng}
\end{flushleft}
\end{minipage}
\begin{minipage}{0.4\textwidth}
\begin{flushright} \large
\emph{} \\
u13057937
\end{flushright}
\end{minipage}
\begin{minipage}{0.4\textwidth}
\begin{flushleft} \large
Antonia {Michael}
\end{flushleft}
\end{minipage}
\begin{minipage}{0.4\textwidth}
\begin{flushright} \large
\emph{} \\
u13014171
\end{flushright}
\end{minipage}
\vfill
% Bottom of the page
{\large \today}
\end{center}
\end{titlepage}
\footnotesize
\documentclass[a4paper,12pt]{report}
\addtolength{\oddsidemargin}{-1.cm}
\addtolength{\textwidth}{2cm}
\addtolength{\topmargin}{-2cm}
\addtolength{\textheight}{3.5cm}
\newcommand{\HRule}{\rule{\linewidth}{0.5mm}}
\makeindex
\setcounter{secnumdepth}{5}

\usepackage{longtable}
\usepackage[utf8]{inputenc}
\usepackage[T1]{fontenc}
\usepackage[pdftex]{graphicx}
\usepackage{makeidx}
\usepackage{hyperref}
\hypersetup{
    colorlinks=true,
    linkcolor=blue,
    filecolor=magenta,      
    urlcolor=cyan,
}


% define the title
\author{Group1_a}
\title{ Assignment 1 Report}
\begin{document}
\setlength{\parskip}{6pt}

% generates the title
\begin{titlepage}

\begin{center}
% Upper part of the page       
\includegraphics[width=1\textwidth]{./up-logo.jpg}\\[0.4cm]    
\textsc{\LARGE Department of Computer Science}\\[1.5cm]
\textsc{\Large COS 301 - Software Engineering}\\[0.5cm]
% Title
\HRule \\[0.4cm]
{ \huge \bfseries COS 301 - Mini Project}\\[0.4cm]
\HRule \\[0.4cm]
% Author and supervisor
\begin{minipage}{0.4\textwidth}
\begin{flushleft} \large
\emph{Author:}\\
Una{Rambani}
\end{flushleft}
\end{minipage}
\begin{minipage}{0.4\textwidth}
\begin{flushright} \large
\emph{Student number:} \\
u14004489
\end{flushright}
\end{minipage}
\begin{minipage}{0.4\textwidth}
\begin{flushleft} \large
kevin {Heritage}
\end{flushleft}
\end{minipage}
\begin{minipage}{0.4\textwidth}
\begin{flushright} \large
\emph{} \\
u13044924
\end{flushright}
\end{minipage}
\begin{minipage}{0.4\textwidth}
\begin{flushleft} \large
Tshepo {Malesela}
\end{flushleft}
\end{minipage}
\begin{minipage}{0.4\textwidth}
\begin{flushright} \large
\emph{} \\
u14211582
\end{flushright}
\end{minipage}
\begin{minipage}{0.4\textwidth}
\begin{flushleft} \large
Vukile {Langa}
\end{flushleft}
\end{minipage}
\begin{minipage}{0.4\textwidth}
\begin{flushright} \large
\emph{} \\
u14035449
\end{flushright}
\end{minipage}
\begin{minipage}{0.4\textwidth}
\begin{flushleft} \large
Wynand {Meirng}
\end{flushleft}
\end{minipage}
\begin{minipage}{0.4\textwidth}
\begin{flushright} \large
\emph{} \\
u13230795
\end{flushright}
\end{minipage}
\begin{minipage}{0.4\textwidth}
\begin{flushleft} \large
Nontokozo {Hlatshwayo}
\end{flushleft}
\end{minipage}
\begin{minipage}{0.4\textwidth}
\begin{flushright} \large
\emph{} \\
u14414555
\end{flushright}
\end{minipage}
\begin{minipage}{0.4\textwidth}
\begin{flushleft} \large
Tim {Kirker}
\end{flushleft}
\end{minipage}
\begin{minipage}{0.4\textwidth}
\begin{flushright} \large
\emph{} \\
u11152402
\end{flushright}
\end{minipage}
\begin{minipage}{0.4\textwidth}
\begin{flushleft} \large
Thabang {Letageng}
\end{flushleft}
\end{minipage}
\begin{minipage}{0.4\textwidth}
\begin{flushright} \large
\emph{} \\
u13057937
\end{flushright}
\end{minipage}
\begin{minipage}{0.4\textwidth}
\begin{flushleft} \large
Antonia {Michael}
\end{flushleft}
\end{minipage}
\begin{minipage}{0.4\textwidth}
\begin{flushright} \large
\emph{} \\
u13014171
\end{flushright}
\end{minipage}
\vfill
% Bottom of the page
{\large \today}
\end{center}
\end{titlepage}
\footnotesize
\input{example.tex}
\normalsize

\renewcommand{\thesection}{\arabic{section}}
\newpage
\begin{center}
\textsc{\LARGE Software Requirements Specification and Technology Neutral Process Design}\\[1.5cm]
\textsc{\Large Buzz Space Discussions/Mini Project}\\[0.5cm]
Further references see \href{https://github.com/ACalitz/COS301Phase2Group1A.git}{gitHub}.
\today
\end{center}
\tableofcontents{}
\newpage
\section{Preface}
		\subsection{Group Members}
			\begin{itemize}
				\item 14004489, Una Rambani�jo
				\item 13044924, Kevin Heritage
				\item 14211582, Tshepo Malesela
				\item 14035449, Vukile Langa
				\item 13230795, Wynand Meiring
				\item 14414555, Nontokozo Hlatshwayo
				\item 11152402, Tim Kirker
				\item 13057937, Thabang Letageng
				\item 13014171, Antonia Michael
			\end{itemize}
		\subsection{GitHub Repository}
			\url{https://github.com/ACalitz/COS301Phase2Group1A.git}
		\subsection{Contributions}
			\subsubsection{Access Channel Requirements}
				\begin{itemize}
					\item Everyone will work on these constraints
				\end{itemize}
			\subsubsection{Quality Requirements}  
				\begin{itemize}
					\item Rendani Dau
					\item Byron Dinkelmann
					\item Antonia Michael
				\end{itemize}
			\subsubsection{Integration Requirements}  
				\begin{itemize}
					\item Izak Blom
					\item Andre Calitz
					\item Chris Cloete
				\end{itemize}
			\subsubsection{Architecture Contraints}
				\begin{itemize}
					\item Daniel Christopher Alves Ara�jo
					\item Tim Kirker
					\item Thabang Letageng
				\end{itemize}
\index{Vision}
\newpage
	\section{Architectural Responsibilities}
	\begin{itemize}
	\item The system must be able to store threads in each buzz space, as well as provide for the creating, updating, deleting and viewing of all threads.
	\item The system must allow multiple users to access the buzz spaces, hence by means of the concurrency the system makes use of.
	\item The system must be scalable, auditable, usable, and reliable.
	\item The system must provide for notification messages to be sent to users if higher level users edit or delete their threads.
	\item The system should provide for multiple deployment.
	\item The system should allow users to up vote and down vote other user's posts as well as comment on them.
	\end{itemize}
\newpage
	\section{Quality Requirements}
	\subsection{Reliability}
	\subsubsection{Reasons for quality requirement}
	\begin{itemize}
	\item This has been given highest priority because Buzz spaces play an important role as users/students will receive important information from buzz spaces and they may be assessed for marks from their contributions to threads on certain buzz spaces. 
	\item To give all students a fair chance and to allow staff (lecturers, Teaching Assistants) to complete their duties (assessing the students) timeously, reliability plays a key role, ensuring that all functions work as the user expects them, when the user requires to use the system.
	\item It must hence have a maximum of an hour down time a day.
	\item Reliability stems further. Being defined as �the ability to be relied upon for accuracy�, one can take into account that the posts on the buzz space have been checked for plagiarism, similarity, netiquette and type of content based on the current status of the user. This hence provides reliability of the system. 
	\item The system is also expected to be reliable in terms of ensuring that users only get the privileges they are entitled to due to their particular status. Hence the system must also reflect an honest account of what rank each student has. 
	\end{itemize}
	\subsubsection{Strategies to achieve this quality requirement}
	\begin{itemize}
	\item Firstly, the prevention of faults. This is done by testing the system thoroughly, using resource locking as well as removing single points of failure. (Solms, 2014)
 \item Secondly, detection of faults, which is achieved through deadlock detection, logging, checkpoint evaluation and error communication to name but a few. Recovering from faults also falls under reliability. This is done by passive redundancy, maintaining backups and checkpoint rollbacks. (Solms, 2014)
 \end{itemize}
 \subsubsection{Patterns to achieve these strategies}
 \begin{itemize}
 \item MVC
 \end{itemize}
 The MVC pattern can  be used for reliability, because since the different layers are clearly separated, hence particular teams are focused on working on each layer, making the system more reliable. 
 \subsection{Auditability/Monitorability}
 \subsubsection{Reasons for quality requirements}
 \begin{itemize}
 \item Given the large user base the system will have, it is paramount that user activity should be track-able and changes made to buzz spaces such as creation of threads, deletion of threads, media uploads, etc. should be traceable back to the person who made these changes.
 \item Also, events that precede system crashes and those of unauthorized users should be traceable.
 \item In the event of a system crash, it should be possible to roll the system back to a previous working state
 \item This also involves high level users such as administrators and lectures being able to follow how
a student/user has participated in discussions, answering questions, asking questions and 
following lecturer input, and also being able to follow statistics such as how many people are active on the system.
 \item This falls under the monitorability aspect of the system. This is a
core quality requirement as students using the system get can get graded on their participation
as discussed above and is therefore a major requirement for the buzz system itself.
 \end{itemize}
 \subsubsection{Strategies to achieve this quality requirement}
 \begin{itemize}
 \item System should have log files running at all times to track all transactions made by users. 
 \item Time stamps should be added to document time and date information of the activities done so that the system can trace through the
information when needed, such as the events that precede a system crash or unauthorized access
that alters the system in any way.
\item System backup should allow rollback when needed.
\item ACID test can be carried out. Acid is an acronym that describes the properties of a database or system. The properties are:
	\begin{itemize}
	\item \textbf{Atomicity:} Defined as all or none situation referring to the processes that take place on the 
	   system. If something where to go wrong with a process such as posting on the system,
	   then the entire process has to be repeated or not at all.
	\item \textbf{Consistency:} All processes must be completed. No process can be left in a half-finished state,
	     if a failure is detected in a process then the entire process has to be rolled back.
	\item \textbf{Isolation:} Keeps process/transactions separate from one another until they are finished.
	\item \textbf{Durability:} The system must keep a backup of its current state so as to roll back to it if
	    the system where to experience a system failure, crash or corruption of data due
	    to a security breach.
	\end{itemize}
\item To ensure monitorability, post metadata, to document user involvement, should be saved. Also, post ranking system to document user quality in discussions.
 \end{itemize}
 \subsubsection{Patterns to achieve these strategies}
 \begin{itemize}
 \item MVC
 \item Layering
\end{itemize} 
 MVC is a suitable pattern because it provides auditability through logging all filter inputs and outputs (off queues). Layering is a suitable pattern because each separate layer can be audited and monitored individually, rather than auditing the system as a whole.
 \subsection{Usability}
 \subsubsection{Reasons for quality requirements}
 \begin{itemize}
 \item Since Buzz spaces may be a source of important information and discussion regarding academics (assignments, practicals etc.), it shouldn't be hard for new users to become familiarized with the system
	\item Especially because the initial user of the buzz space will indeed be a first year student who has probably had minimal exposure to discussion boards and how to use them. The system should also be rememberable hence it must also be understandable. 
	\item It should take at most 3 hours for an average user to be familiar with the system
 \end{itemize}
 \subsubsection{Strategies to achieve this quality requirement}
 \begin{itemize}
 \item Various goals of usability requirements are firstly, that the that the interface is intuitive, i.e. easy to navigate and understand, that the buttons and icons are self explanatory for the primary users.
 \item The interface must also not be a cluttered, frustrating and overwhelming one. 
 \item Ease of learning is also an important goal here such that users who have never used such a system can catch on easily and such that users who regularly use other discussion boards will not get confused and displaced. 
 \item The system must also be task efficient, i.e. if users access this space regularly, long tedious processes and other admin must be avoided.
\item Also, the colour schemes, functionality and interactiveness of the interface and system must contribute to this task efficiency. 
\item Different usability tests can be conducted such as handing out paper prototypes of different interface designs, and questionnaires getting feedback from the sample of people that were consulted in the survey. Problems with the different interfaces can be picked up during the usability testing phase, as indicated by the sample of users consulted, such that the final product will be much more user friendly. (http://www.usability.gov/what-and-why/usability-evaluation.html)
 \end{itemize}
 \subsubsection{Patterns to achieve these strategies}
 \begin{itemize}
 \item MVC
 \item Layering
\end{itemize}
MVC is a suitable pattern because the user will only need to interact with the front end interface, rather than dealing with the technical aspects of the back-end system. Another reason for this is that the developers allocated to working on the View will have the sole focus of making it usable.
Layering can be used within the subsections of MVC, i.e. the Control and Model layer can be layered to further divide concerns and allow different people to work on those layers.

\subsection{Scalability}
	\subsubsection{Reasons for quality requirement}
	\begin{itemize}
	\item This is a core requirement mainly because of the volume of students that will be accessing this discussion board i.e. all third/fourth year undergraduate students.
	\item Each of these students have approximately four COS modules in each year, hence this system will need to cater to this large mass of students as well as the lecturers, tutors and admin staff.
	\item With this, we can assume that there will be in excess of 2000 users meaning that the system has to have the ability to handle at least 1000 concurrent users at peak times.
	\item Different Buzz Spaces can have any number of threads which in turn may have resources (BuzzResources) such as media (videos, pod-casts and images) and documents (pdf, odf, doc, etc.) associated with them. This means that the system has to be able to manage the expansion of storage resources (mainly HDD's on the server)
	\end{itemize}
	\subsubsection{Strategies to achieve this quality requirement}
	\begin{itemize}
		\item We will need to firstly ensure that existing resources are managed efficiently, i.e. reducing the load using efficient storage, processing,  and persistence. In addition, we will need to ensure that the load is spread across resources and time, using methods of load balancing to spread load across resources as well as using scheduling and queueing to spread load across time.
		\item Secondly, the resources can be scaled up by increasing storage, increasing processing power and increasing the capacity of communication channels.
		\item Lastly, resources can be scaled out by means of using external resources, using commoditized resources and distributing tasks across specialized resources.
		\end{itemize}
	\subsubsection{Patterns to achieve these strategies}
	 \begin{itemize}
		\item Concurrency Master-Slave 
	\end{itemize}
We chose this pattern here due to the concurrency of the system, meaning that a large number of users must be able to access the system at a time.
 
 \subsection{Integrability}
 \subsubsection{Reasons for quality requirements}
 \begin{itemize}
 \item The buzz system is not a stand alone system as it requires an external database and website to integrate into. For example the LDAP Database, and Computer Science website.
\item The system must be portable to other system platforms so that it can be adapted to other client specific applications. For example another universities database and website.
\item It should take at most 1 day to integrate the system into another system.  
 \end{itemize}
 \subsubsection{Strategies to achieve this quality requirement}
 \begin{itemize}
 \item The interface must be structured between the system and database so that queries can be handled by any database with minimal changes to the interface to do so.
 \item The interface must be structured to be independent of external HTML and scripting languages so that it can be seamlessly integrated into any website or with minimal changes to either the system or website. 
 \end{itemize}
 \subsubsection{Patterns to achieve these strategies}
 \begin{itemize}
 \item MVC
 \end{itemize}
MVC is a sufficient pattern to use here, because firstly it is simplifies the systems through separation of concerns. Also, because it improves on reuse, the system can be integrated into other systems without having to create entirely new components.

\subsection{Nice to have}
\begin{itemize}
	\item Maintainability
	\item Flexibility
	\item Performance
	\item Security
\end{itemize}

\newpage
	\section{Integration Requirements} 
	\subsection{Integration channels}
	\subsubsection{LDAP Database System}
	LDAP, the database used by the computer science department. Integration with this system is required for retrieval of the majority of the information that the buzz system will require. Performance for this channel is also critical and thus we 				recommend the ftp protocol as it will allow quick access to the system so that all the 			required information can be obtained as fast as possible. The reduced security of this protocol is negligible as the connections will all be local on the same server and OpenSSH can be used for additional security (See FTP protocol discussion below.
	\paragraph{Quality Requirements for the Database System}
	\begin{itemize}
	\item{\textbf{Performance:} All queries to the LDAP database should be rapid and efficient to provide the best user experience to the maximum amount of people. This can be achieved through interacting with the system using the ftp protocol since the Buzz system will be hosted on the same servers as the LDAP system.}
	\item{\textbf{Reliability:} This is also an important requirement as the Buzz system needs to have as little downtime as possible and thus a reliable connection with the LDAP server is required. }
	\item{\textbf{Scalability:} The system needs to be able to work with a large amount of users concurrently.}
	\item{\textbf{Integrability:} The integration with the LDAP database should be engineered in a manner which can easily be adjusted to accommodate additional integrations as the LDAP system evolves.}
	\item{\textbf{Affordability:} Access or queries made to the LDAP database should be affordable. That is, it should be possible to make requests as often as needed without any implications on the overall system performance.}
	\item{\textbf{Flexibility:} The integration with the LDAP database should be flexible in the sense that slight changes in the database functionality should not affect the integration with the Buzz System.}
	\item{\textbf{Auditability:} All information changes, user interactions, database queries should be auditable. i.e. Identification, time-stamps etcetera should be linked to every action with regards to the integration channels to be used. Every integration action must be traceable to a user or system action.}
	\end{itemize}

  \subsubsection{Web services}
  Web services namely HTTP and TCP, will facilitate user interaction with the Buzz system. 
	\paragraph{Quality Requirements for the Web Services}
	\begin{itemize}
	\item{\textbf{Security:} This is an important quality requirement in any web-based application that incorporates authentication and the storage of personal information.}
	\item{\textbf{Reliability:} This is also an important requirement as the Buzz system needs to have as little downtime as possible and Web service integrations should be functional at all times. }
	\item{\textbf{Integrability:} Is of importance as this service needs to integrate with a host of other services that provide some other form of functionality to the buzz system, be it the notification system, authentication system etc.}
	\item{\textbf{Performance:} The performance of these integration channels should be a priority to avoid lowering the overall Buzz system performance. This negatively implicates the use of https which performs somewhat slower than http, but the increased security provides enough motivation to make use of https.}
	\item{\textbf{Scalability:} The protocols used need to be competent to address at least 10 000 requests per second for example to ensure optimal system performance and prevent breakdown.}
	\end{itemize}
\subsection{Access channels}
	Buzz has various access channels from which users can gain access to the system:
	\begin{itemize}
		\item All users of Buzz (mostly students, but will still be accessed by lecturers, administrators, etc.) will have access to the system via the Computer Science website. Users will use a modern web browser, such as Mozilla Firefox, Google Chrome, Microsoft Internet Explorer, or Apple Safari to interact with the system. Students and lecturers (or administrators) will, however, have different permissions. Thus, the interface will be altered depending on the permissions the user has.
		\item Users will also have access to Buzz via an Android application.
	\end{itemize} 	


\subsection{Protocols}
\subsubsection{Database Query Language}
	SQL will be used to query the LDAP database by making use of the PostgreSQL database management system. \textbf{This is not classified as a protocol, but is mentioned here for the sake of clarity.}
\subsubsection{HTTP - Hypertext Transfer Protocol}
Integration with this protocol will occur at a high level and typically be handled by libraries or browser-clients etc.
\textbf{To be used for:	}
	\begin{itemize}
	\item{All data transferred between users and the server on which the system is hosted.}
	\item{All data transferred between the system and LDAP.}
	\item{Transfer of miscellaneous data such as HTTP error codes to ensure both servers and clients are aware of the state of data transfers and its results}
	\end{itemize}
\subsubsection{TCP - Transmission Control Protocol}
 For establishing network connections between the user computers and the system server as well as between the system server and the LDAP server. Streams of data can then be exchanged between the connected hosts. Error detection, faulty transmission of data, resending of data etc. will all be done using TCP (Davids). Integration with this protocol will occur at a high level and will typically be handled by libraries or operating system functions.
\subsubsection{FTP - File Transfer Protocol}
To enhance system performance, FTP will be used for data transfer between the LDAP database and the Buzz system wherever possible, since both systems will possibly be hosted on the same server and therefore the lowered security (nurdletech.com) of the FTP protocol is negligible to some extent. OpenSSH (Open Secure Shell) can alternatively used to secure FTP connections.
\subsubsection{SMTP - Simple Mail Transfer Protocol}
This protocol will be used to handle e-mail communication related to the Buzz system. It addresses Security as a quality requirement since it incorporates SMTP-Authentication defined by RFC 2554 (Meyers, 1999) which enhances the security of the protocol.

\subsection{API specifications}
\subsubsection{Apache Maven}
Apache Maven this can be used to help with the build process of the system as it will allow for easy integration with other external services.
\subsubsection{WSDL - Web Services Description Language}
WSDL can be used for information exchange between systems as it provides an effective means of sending messages over the network.
\subsubsection{JPA - Java Persistence API}
JPA can be used if a java database is used.
\subsubsection{PostgreSQL}
PostgreSQL can be used as an alternative means for interacting with a database in the case that a java based database is not used.
\subsubsection{GIT}
GIT can be used as a version control API as it will allow an easy means to manage the versions and also solve code conflicts.


\newpage
	\section{Architecture Constraints}
		\subsection{Reference Architecture}
			Buzz's system architecture will be using Java EE as a reference architecture, as specified by the client. Reasons for the use of Java EE is:
			\begin{itemize}
				\item It is a set of standard specifications and is  therefore independent of any particular vendor. Often, there are a number of implementations of the Java EE specifications. (W�hner, n.d.).
				\item Sustainability. (W�hner, n.d.).
				\item Portability (projects are easy to migrate to Java EE). (Bien, 2009).
				\item Java EE implementations are lightweight. (Bien, 2009) (W�hner, n.d.).
			\end{itemize}
		\subsection{Technologies}
			The primary programming language used for Buzz is \textbf{Java}.
			A few additional technologies will be used in the development of Buzz:
			\begin{itemize}
				\item \textbf{Apache Maven}: Maven is a project management and build tool for Java. (http://maven.apache.org/).
				\item \textbf{Git and GitHub}: A version control software and a repository website that will be used to host the source code of Buzz. Reasons for the use of Git is its ease of use and because it is free and open source. (http://git-scm.com/).
				\item \textbf{JPA (Java Persistence API)}: It is an API for Java that will describe the management of relational data. (Solms, 2015).
				\item \textbf{JSF (JavaServer Faces)}: It is a specification for building server-based user interfaces. (http://www.oracle.com).\\ Several reasons for using JSF, including the ability to define a page using HTML and the ease of composing custom, reusable components. (Borges, 2013).
				\item \textbf{PostgreSQL}: PostgreSQL is a free, open-source, cross-platform, object-orientated database management system. (http://www.postgresql.org/about/).\\
				The reason we chose it is because of its unlimited maximum database size, its compliance with ANSI-SQL and features like Multi-Version Concurrency Control. (http://www.postgresql.org/about/).\\
				The reason for using PostgreSQL rather than Microsoft SQL Server Express is because it has restrictions, such as a maximum database size of 10GB (http://blogs.msdn.com), which the authors believe may be too limited.
				\item \textbf{JPQL (Java Persistence Query Language)}: JPQL is a technology-neutral object-orientated query language used to "formulate queries across object graphs." (Solms, 2015)
			\end{itemize}
		\subsection{Operating Systems}
			Buzz will be designed to run on a Linux-based operating system. Linux-based operating systems are free, open-source and provide a stable base for the system. (Beal, n.d.).\\
			\\
			A Google Android client will be designed for Buzz that will allow users to log into and interact with Buzz on their phones. However, considering that Buzz will be linked to the Computer Science website, it will still be accessible with a mobile browser (for example, Chrome for Android, Firefox for Android, etc.). iOS devices will still be able to access Buzz through Safari.\\
			\\
			Buzz will be able to run on almost all operating systems that runs a modern web browser like Mozilla Firefox or Google Chrome (for example, Microsoft Windows, Apple Mac OSX or any distribution of Linux or BSD).
	\section{Architectural Patterns}
		For the design of Buzz, two patterns are considered: the \textbf{MVC (Model-View-Controller) pattern} and the \textbf{Layering architectural pattern}. For concurrency, the \textbf{Master-Slave pattern} is used.\\
		The top layers of the Model and the Controller are interfaces for the MVC architecture and the layers below provide the functionality.\\
		MVC is considered because:
		\begin{itemize}
			\item It provides modularity (i.e. the system's concerns are separated, thus easier to implement). (Solms, 2014)
			\item It allows for better maintainability (one can maintain the Model, View and Controller separately). (Solms, 2014)
			\item Testability (it is easier to test because of separated concerns, so the source of any problems are easy to identify). (Solms, 2014)
			\item Reuse (it is possible to take any component and reuse it where necessary). (Solms, 2014)
		\end{itemize} 
		Layering allows Buzz to have pluggable layers, which will allow the developers to replace layers as needed.
		This pattern allows for: 
		\begin{itemize}
			\item Improved cohesion. (Solms, 2014)
			\item Reduced complexity of the system. (Solms, 2014)
			\item Improved testability (which will allow for easier debugging). (Solms, 2014)
			\item Improved reuse and maintainability of the source code, because all the layers are individual and separate from one another. (Solms, 2014)
		\end{itemize}
		However, it should be mentioned that Layering has a performance overhead associated with it, as well as higher maintenance costs associated with the lower layers, because they impact the higher levels. (Solms, 2014). Given the benefits, however, the authors feel the reduced performance and maintenance costs is a good compromise for reduced complexity and testability.\\
		\\
		For concurrency, the Master-Slave architectural pattern (Solms, 2014) is considered, because the system needs to accommodate a large number of users at a time.
	\newpage
	\section{Bibliography}
		\begin{itemize}
			\item \textit{Basic MVC Architecture}. [Online]. Available: <http://www.tutorialspoint.com/struts\_2/basic\\\_mvc\_architecture.htm> [Accessed 5 March 2015].
			
			\item Beal, V. n.d. \textit{Linux Server}. [Online]. Available: <http://www.webopedia.com/TERM/L/linux\_\\server.html> [Accessed 10 March 2015].

			\item Bien, A. 2009. \textit{9 Reasons Why Java EE 6 Will Save Real Money - Or How To Convince Your Management}. [Online]. Available: <http://www.adam-bien.com/roller/abien/entry/\\8\_reasons\_why\_java\_ee> [Accessed 10 March 2015].

			\item Borges, B. 2013. \textit{Reasons to why I'm reconsidering JSF}. [Online]. Available: <http://blog.brunoborges.com.br/2013/01/reasons-to-why-im-reconsidering-jsf.html> [Accessed 10 March 2015].
			
			\item Davids, N.\textit{The Limitations of the Ethernet CRC and TCP/IP checksums for error detection} [Online]. Available: 
			<http://noahdavids.org/self-published/CRC-and-checksum.html>[Accessed 8 March 2015].

			\item \textit{Git --loval-branding-on-the-cheap}. [Online]. Available: <http://git-scm.com/> [Accessed 10 March 2015].

			\item \textit{JavaServer Faces Technology}. [Online]. Available: <http://www.oracle.com/technetwork/java/javaee/javaserverfaces-139869.html> [Accessed 10 March 2015].

			\item Kabanov, J. 2011. \textit{Ed Burns on Why JSF is the Most Popular Framework}. [Online]. Available: <http://zeroturnaround.com/rebellabs/ed-burns-on-why-jsf-is-the-most-popular-framework/> [Accessed 10 March 2015].
			\item Meyers, J. 1999. \textit{SMTP Service Extension for Authentication} [Online]. Available: <http://tools.ietf.org/html/rfc2554> [Accessed 9 March 2015]

			\item \textit{Securing FTP using SSH} [Online]. Available: <http://nurdletech.com/linux-notes/ftp/ssh.html> [Accessed 9 March 2015]

			\item \textit{PostgreSQL: About}. [Online]. Available: <http://www.postgresql.org/about/> [Accessed 10 March 2015].
			
			\item \textit{Software Engineering}. [Online]. Available: <http://sesolution.blogspot.com/p/software-engineering-layered-technology.html>[Accessed 5 March 2015].
								
			\item Solms, F. 2014. \textit{Software Architecture Desgin.} [Online]. University of Pretoria: Pretoria. Available: <http://www.cs.up.ac.za/modules/courses/download.php?id=8565> [Accessed 9 March 2015].
			
			\item Solms, F. 2015. \textit{Java Persistence API (JPA).} [Online]. University of Pretoria: Pretoria. Available: <http://www.cs.up.ac.za/modules/courses/download.php?id=8640> [Accessed 7 March 2015].
			
			\item \textit{SQL Server 2008 R2 Express Database Size Limit Increased to 10GB}. [Online]. Available: <http://blogs.msdn.com/b/sqlexpress/archive/2010/04/21/database-size-limit-increased-to-10gb-in-sql-server-2008-r2-express.aspx> [Accessed 9 March 2015].
			
			\item \textit{Usability Evaluation Basics}. [Online]. Available: <http://www.usability.gov/what-and-why/usability-evaluation.html> [Accessed 5 March 2015].

			\item W�hner, K. n.d. \textit{Why I will use Java EE (JEE, and not J2EE) instead of Spring in new Enterprise Java Projects in 2012}. [Online]. Available: <www.kai-waehner.de/blog/2011/11/21/why-i-will-use-java-ee-jee-and-not-j2ee-instead-of-spring-in-new-enterprise-java-projects-in-2012/> [Accessed 10 March 2015].		

			\item \textit{Welcome to Apache Maven}. [Online]. Available: <http://maven.apache.org/> [Accessed 10 March 2015].
			

			

			

			

			

			

		\end{itemize}
\bibliography{myrefs}{} 
\bibliographystyle{ieeetr}
\end{document}
\normalsize

\renewcommand{\thesection}{\arabic{section}}
\newpage
\begin{center}
\textsc{\LARGE Software Requirements Specification and Technology Neutral Process Design}\\[1.5cm]
\textsc{\Large Buzz Space Discussions/Mini Project}\\[0.5cm]
Further references see \href{https://github.com/ACalitz/COS301Phase2Group1A.git}{gitHub}.
\today
\end{center}
\tableofcontents{}
\newpage
\section{Preface}
		\subsection{Group Members}
			\begin{itemize}
				\item 14004489, Una Rambani�jo
				\item 13044924, Kevin Heritage
				\item 14211582, Tshepo Malesela
				\item 14035449, Vukile Langa
				\item 13230795, Wynand Meiring
				\item 14414555, Nontokozo Hlatshwayo
				\item 11152402, Tim Kirker
				\item 13057937, Thabang Letageng
				\item 13014171, Antonia Michael
			\end{itemize}
		\subsection{GitHub Repository}
			\url{https://github.com/ACalitz/COS301Phase2Group1A.git}
		\subsection{Contributions}
			\subsubsection{Access Channel Requirements}
				\begin{itemize}
					\item Everyone will work on these constraints
				\end{itemize}
			\subsubsection{Quality Requirements}  
				\begin{itemize}
					\item Rendani Dau
					\item Byron Dinkelmann
					\item Antonia Michael
				\end{itemize}
			\subsubsection{Integration Requirements}  
				\begin{itemize}
					\item Izak Blom
					\item Andre Calitz
					\item Chris Cloete
				\end{itemize}
			\subsubsection{Architecture Contraints}
				\begin{itemize}
					\item Daniel Christopher Alves Ara�jo
					\item Tim Kirker
					\item Thabang Letageng
				\end{itemize}
\index{Vision}
\newpage
	\section{Architectural Responsibilities}
	\begin{itemize}
	\item The system must be able to store threads in each buzz space, as well as provide for the creating, updating, deleting and viewing of all threads.
	\item The system must allow multiple users to access the buzz spaces, hence by means of the concurrency the system makes use of.
	\item The system must be scalable, auditable, usable, and reliable.
	\item The system must provide for notification messages to be sent to users if higher level users edit or delete their threads.
	\item The system should provide for multiple deployment.
	\item The system should allow users to up vote and down vote other user's posts as well as comment on them.
	\end{itemize}
\newpage
	\section{Quality Requirements}
	\subsection{Reliability}
	\subsubsection{Reasons for quality requirement}
	\begin{itemize}
	\item This has been given highest priority because Buzz spaces play an important role as users/students will receive important information from buzz spaces and they may be assessed for marks from their contributions to threads on certain buzz spaces. 
	\item To give all students a fair chance and to allow staff (lecturers, Teaching Assistants) to complete their duties (assessing the students) timeously, reliability plays a key role, ensuring that all functions work as the user expects them, when the user requires to use the system.
	\item It must hence have a maximum of an hour down time a day.
	\item Reliability stems further. Being defined as �the ability to be relied upon for accuracy�, one can take into account that the posts on the buzz space have been checked for plagiarism, similarity, netiquette and type of content based on the current status of the user. This hence provides reliability of the system. 
	\item The system is also expected to be reliable in terms of ensuring that users only get the privileges they are entitled to due to their particular status. Hence the system must also reflect an honest account of what rank each student has. 
	\end{itemize}
	\subsubsection{Strategies to achieve this quality requirement}
	\begin{itemize}
	\item Firstly, the prevention of faults. This is done by testing the system thoroughly, using resource locking as well as removing single points of failure. (Solms, 2014)
 \item Secondly, detection of faults, which is achieved through deadlock detection, logging, checkpoint evaluation and error communication to name but a few. Recovering from faults also falls under reliability. This is done by passive redundancy, maintaining backups and checkpoint rollbacks. (Solms, 2014)
 \end{itemize}
 \subsubsection{Patterns to achieve these strategies}
 \begin{itemize}
 \item MVC
 \end{itemize}
 The MVC pattern can  be used for reliability, because since the different layers are clearly separated, hence particular teams are focused on working on each layer, making the system more reliable. 
 \subsection{Auditability/Monitorability}
 \subsubsection{Reasons for quality requirements}
 \begin{itemize}
 \item Given the large user base the system will have, it is paramount that user activity should be track-able and changes made to buzz spaces such as creation of threads, deletion of threads, media uploads, etc. should be traceable back to the person who made these changes.
 \item Also, events that precede system crashes and those of unauthorized users should be traceable.
 \item In the event of a system crash, it should be possible to roll the system back to a previous working state
 \item This also involves high level users such as administrators and lectures being able to follow how
a student/user has participated in discussions, answering questions, asking questions and 
following lecturer input, and also being able to follow statistics such as how many people are active on the system.
 \item This falls under the monitorability aspect of the system. This is a
core quality requirement as students using the system get can get graded on their participation
as discussed above and is therefore a major requirement for the buzz system itself.
 \end{itemize}
 \subsubsection{Strategies to achieve this quality requirement}
 \begin{itemize}
 \item System should have log files running at all times to track all transactions made by users. 
 \item Time stamps should be added to document time and date information of the activities done so that the system can trace through the
information when needed, such as the events that precede a system crash or unauthorized access
that alters the system in any way.
\item System backup should allow rollback when needed.
\item ACID test can be carried out. Acid is an acronym that describes the properties of a database or system. The properties are:
	\begin{itemize}
	\item \textbf{Atomicity:} Defined as all or none situation referring to the processes that take place on the 
	   system. If something where to go wrong with a process such as posting on the system,
	   then the entire process has to be repeated or not at all.
	\item \textbf{Consistency:} All processes must be completed. No process can be left in a half-finished state,
	     if a failure is detected in a process then the entire process has to be rolled back.
	\item \textbf{Isolation:} Keeps process/transactions separate from one another until they are finished.
	\item \textbf{Durability:} The system must keep a backup of its current state so as to roll back to it if
	    the system where to experience a system failure, crash or corruption of data due
	    to a security breach.
	\end{itemize}
\item To ensure monitorability, post metadata, to document user involvement, should be saved. Also, post ranking system to document user quality in discussions.
 \end{itemize}
 \subsubsection{Patterns to achieve these strategies}
 \begin{itemize}
 \item MVC
 \item Layering
\end{itemize} 
 MVC is a suitable pattern because it provides auditability through logging all filter inputs and outputs (off queues). Layering is a suitable pattern because each separate layer can be audited and monitored individually, rather than auditing the system as a whole.
 \subsection{Usability}
 \subsubsection{Reasons for quality requirements}
 \begin{itemize}
 \item Since Buzz spaces may be a source of important information and discussion regarding academics (assignments, practicals etc.), it shouldn't be hard for new users to become familiarized with the system
	\item Especially because the initial user of the buzz space will indeed be a first year student who has probably had minimal exposure to discussion boards and how to use them. The system should also be rememberable hence it must also be understandable. 
	\item It should take at most 3 hours for an average user to be familiar with the system
 \end{itemize}
 \subsubsection{Strategies to achieve this quality requirement}
 \begin{itemize}
 \item Various goals of usability requirements are firstly, that the that the interface is intuitive, i.e. easy to navigate and understand, that the buttons and icons are self explanatory for the primary users.
 \item The interface must also not be a cluttered, frustrating and overwhelming one. 
 \item Ease of learning is also an important goal here such that users who have never used such a system can catch on easily and such that users who regularly use other discussion boards will not get confused and displaced. 
 \item The system must also be task efficient, i.e. if users access this space regularly, long tedious processes and other admin must be avoided.
\item Also, the colour schemes, functionality and interactiveness of the interface and system must contribute to this task efficiency. 
\item Different usability tests can be conducted such as handing out paper prototypes of different interface designs, and questionnaires getting feedback from the sample of people that were consulted in the survey. Problems with the different interfaces can be picked up during the usability testing phase, as indicated by the sample of users consulted, such that the final product will be much more user friendly. (http://www.usability.gov/what-and-why/usability-evaluation.html)
 \end{itemize}
 \subsubsection{Patterns to achieve these strategies}
 \begin{itemize}
 \item MVC
 \item Layering
\end{itemize}
MVC is a suitable pattern because the user will only need to interact with the front end interface, rather than dealing with the technical aspects of the back-end system. Another reason for this is that the developers allocated to working on the View will have the sole focus of making it usable.
Layering can be used within the subsections of MVC, i.e. the Control and Model layer can be layered to further divide concerns and allow different people to work on those layers.

\subsection{Scalability}
	\subsubsection{Reasons for quality requirement}
	\begin{itemize}
	\item This is a core requirement mainly because of the volume of students that will be accessing this discussion board i.e. all third/fourth year undergraduate students.
	\item Each of these students have approximately four COS modules in each year, hence this system will need to cater to this large mass of students as well as the lecturers, tutors and admin staff.
	\item With this, we can assume that there will be in excess of 2000 users meaning that the system has to have the ability to handle at least 1000 concurrent users at peak times.
	\item Different Buzz Spaces can have any number of threads which in turn may have resources (BuzzResources) such as media (videos, pod-casts and images) and documents (pdf, odf, doc, etc.) associated with them. This means that the system has to be able to manage the expansion of storage resources (mainly HDD's on the server)
	\end{itemize}
	\subsubsection{Strategies to achieve this quality requirement}
	\begin{itemize}
		\item We will need to firstly ensure that existing resources are managed efficiently, i.e. reducing the load using efficient storage, processing,  and persistence. In addition, we will need to ensure that the load is spread across resources and time, using methods of load balancing to spread load across resources as well as using scheduling and queueing to spread load across time.
		\item Secondly, the resources can be scaled up by increasing storage, increasing processing power and increasing the capacity of communication channels.
		\item Lastly, resources can be scaled out by means of using external resources, using commoditized resources and distributing tasks across specialized resources.
		\end{itemize}
	\subsubsection{Patterns to achieve these strategies}
	 \begin{itemize}
		\item Concurrency Master-Slave 
	\end{itemize}
We chose this pattern here due to the concurrency of the system, meaning that a large number of users must be able to access the system at a time.
 
 \subsection{Integrability}
 \subsubsection{Reasons for quality requirements}
 \begin{itemize}
 \item The buzz system is not a stand alone system as it requires an external database and website to integrate into. For example the LDAP Database, and Computer Science website.
\item The system must be portable to other system platforms so that it can be adapted to other client specific applications. For example another universities database and website.
\item It should take at most 1 day to integrate the system into another system.  
 \end{itemize}
 \subsubsection{Strategies to achieve this quality requirement}
 \begin{itemize}
 \item The interface must be structured between the system and database so that queries can be handled by any database with minimal changes to the interface to do so.
 \item The interface must be structured to be independent of external HTML and scripting languages so that it can be seamlessly integrated into any website or with minimal changes to either the system or website. 
 \end{itemize}
 \subsubsection{Patterns to achieve these strategies}
 \begin{itemize}
 \item MVC
 \end{itemize}
MVC is a sufficient pattern to use here, because firstly it is simplifies the systems through separation of concerns. Also, because it improves on reuse, the system can be integrated into other systems without having to create entirely new components.

\subsection{Nice to have}
\begin{itemize}
	\item Maintainability
	\item Flexibility
	\item Performance
	\item Security
\end{itemize}

\newpage
	\section{Integration Requirements} 
	\subsection{Integration channels}
	\subsubsection{LDAP Database System}
	LDAP, the database used by the computer science department. Integration with this system is required for retrieval of the majority of the information that the buzz system will require. Performance for this channel is also critical and thus we 				recommend the ftp protocol as it will allow quick access to the system so that all the 			required information can be obtained as fast as possible. The reduced security of this protocol is negligible as the connections will all be local on the same server and OpenSSH can be used for additional security (See FTP protocol discussion below.
	\paragraph{Quality Requirements for the Database System}
	\begin{itemize}
	\item{\textbf{Performance:} All queries to the LDAP database should be rapid and efficient to provide the best user experience to the maximum amount of people. This can be achieved through interacting with the system using the ftp protocol since the Buzz system will be hosted on the same servers as the LDAP system.}
	\item{\textbf{Reliability:} This is also an important requirement as the Buzz system needs to have as little downtime as possible and thus a reliable connection with the LDAP server is required. }
	\item{\textbf{Scalability:} The system needs to be able to work with a large amount of users concurrently.}
	\item{\textbf{Integrability:} The integration with the LDAP database should be engineered in a manner which can easily be adjusted to accommodate additional integrations as the LDAP system evolves.}
	\item{\textbf{Affordability:} Access or queries made to the LDAP database should be affordable. That is, it should be possible to make requests as often as needed without any implications on the overall system performance.}
	\item{\textbf{Flexibility:} The integration with the LDAP database should be flexible in the sense that slight changes in the database functionality should not affect the integration with the Buzz System.}
	\item{\textbf{Auditability:} All information changes, user interactions, database queries should be auditable. i.e. Identification, time-stamps etcetera should be linked to every action with regards to the integration channels to be used. Every integration action must be traceable to a user or system action.}
	\end{itemize}

  \subsubsection{Web services}
  Web services namely HTTP and TCP, will facilitate user interaction with the Buzz system. 
	\paragraph{Quality Requirements for the Web Services}
	\begin{itemize}
	\item{\textbf{Security:} This is an important quality requirement in any web-based application that incorporates authentication and the storage of personal information.}
	\item{\textbf{Reliability:} This is also an important requirement as the Buzz system needs to have as little downtime as possible and Web service integrations should be functional at all times. }
	\item{\textbf{Integrability:} Is of importance as this service needs to integrate with a host of other services that provide some other form of functionality to the buzz system, be it the notification system, authentication system etc.}
	\item{\textbf{Performance:} The performance of these integration channels should be a priority to avoid lowering the overall Buzz system performance. This negatively implicates the use of https which performs somewhat slower than http, but the increased security provides enough motivation to make use of https.}
	\item{\textbf{Scalability:} The protocols used need to be competent to address at least 10 000 requests per second for example to ensure optimal system performance and prevent breakdown.}
	\end{itemize}
\subsection{Access channels}
	Buzz has various access channels from which users can gain access to the system:
	\begin{itemize}
		\item All users of Buzz (mostly students, but will still be accessed by lecturers, administrators, etc.) will have access to the system via the Computer Science website. Users will use a modern web browser, such as Mozilla Firefox, Google Chrome, Microsoft Internet Explorer, or Apple Safari to interact with the system. Students and lecturers (or administrators) will, however, have different permissions. Thus, the interface will be altered depending on the permissions the user has.
		\item Users will also have access to Buzz via an Android application.
	\end{itemize} 	


\subsection{Protocols}
\subsubsection{Database Query Language}
	SQL will be used to query the LDAP database by making use of the PostgreSQL database management system. \textbf{This is not classified as a protocol, but is mentioned here for the sake of clarity.}
\subsubsection{HTTP - Hypertext Transfer Protocol}
Integration with this protocol will occur at a high level and typically be handled by libraries or browser-clients etc.
\textbf{To be used for:	}
	\begin{itemize}
	\item{All data transferred between users and the server on which the system is hosted.}
	\item{All data transferred between the system and LDAP.}
	\item{Transfer of miscellaneous data such as HTTP error codes to ensure both servers and clients are aware of the state of data transfers and its results}
	\end{itemize}
\subsubsection{TCP - Transmission Control Protocol}
 For establishing network connections between the user computers and the system server as well as between the system server and the LDAP server. Streams of data can then be exchanged between the connected hosts. Error detection, faulty transmission of data, resending of data etc. will all be done using TCP (Davids). Integration with this protocol will occur at a high level and will typically be handled by libraries or operating system functions.
\subsubsection{FTP - File Transfer Protocol}
To enhance system performance, FTP will be used for data transfer between the LDAP database and the Buzz system wherever possible, since both systems will possibly be hosted on the same server and therefore the lowered security (nurdletech.com) of the FTP protocol is negligible to some extent. OpenSSH (Open Secure Shell) can alternatively used to secure FTP connections.
\subsubsection{SMTP - Simple Mail Transfer Protocol}
This protocol will be used to handle e-mail communication related to the Buzz system. It addresses Security as a quality requirement since it incorporates SMTP-Authentication defined by RFC 2554 (Meyers, 1999) which enhances the security of the protocol.

\subsection{API specifications}
\subsubsection{Apache Maven}
Apache Maven this can be used to help with the build process of the system as it will allow for easy integration with other external services.
\subsubsection{WSDL - Web Services Description Language}
WSDL can be used for information exchange between systems as it provides an effective means of sending messages over the network.
\subsubsection{JPA - Java Persistence API}
JPA can be used if a java database is used.
\subsubsection{PostgreSQL}
PostgreSQL can be used as an alternative means for interacting with a database in the case that a java based database is not used.
\subsubsection{GIT}
GIT can be used as a version control API as it will allow an easy means to manage the versions and also solve code conflicts.


\newpage
	\section{Architecture Constraints}
		\subsection{Reference Architecture}
			Buzz's system architecture will be using Java EE as a reference architecture, as specified by the client. Reasons for the use of Java EE is:
			\begin{itemize}
				\item It is a set of standard specifications and is  therefore independent of any particular vendor. Often, there are a number of implementations of the Java EE specifications. (W�hner, n.d.).
				\item Sustainability. (W�hner, n.d.).
				\item Portability (projects are easy to migrate to Java EE). (Bien, 2009).
				\item Java EE implementations are lightweight. (Bien, 2009) (W�hner, n.d.).
			\end{itemize}
		\subsection{Technologies}
			The primary programming language used for Buzz is \textbf{Java}.
			A few additional technologies will be used in the development of Buzz:
			\begin{itemize}
				\item \textbf{Apache Maven}: Maven is a project management and build tool for Java. (http://maven.apache.org/).
				\item \textbf{Git and GitHub}: A version control software and a repository website that will be used to host the source code of Buzz. Reasons for the use of Git is its ease of use and because it is free and open source. (http://git-scm.com/).
				\item \textbf{JPA (Java Persistence API)}: It is an API for Java that will describe the management of relational data. (Solms, 2015).
				\item \textbf{JSF (JavaServer Faces)}: It is a specification for building server-based user interfaces. (http://www.oracle.com).\\ Several reasons for using JSF, including the ability to define a page using HTML and the ease of composing custom, reusable components. (Borges, 2013).
				\item \textbf{PostgreSQL}: PostgreSQL is a free, open-source, cross-platform, object-orientated database management system. (http://www.postgresql.org/about/).\\
				The reason we chose it is because of its unlimited maximum database size, its compliance with ANSI-SQL and features like Multi-Version Concurrency Control. (http://www.postgresql.org/about/).\\
				The reason for using PostgreSQL rather than Microsoft SQL Server Express is because it has restrictions, such as a maximum database size of 10GB (http://blogs.msdn.com), which the authors believe may be too limited.
				\item \textbf{JPQL (Java Persistence Query Language)}: JPQL is a technology-neutral object-orientated query language used to "formulate queries across object graphs." (Solms, 2015)
			\end{itemize}
		\subsection{Operating Systems}
			Buzz will be designed to run on a Linux-based operating system. Linux-based operating systems are free, open-source and provide a stable base for the system. (Beal, n.d.).\\
			\\
			A Google Android client will be designed for Buzz that will allow users to log into and interact with Buzz on their phones. However, considering that Buzz will be linked to the Computer Science website, it will still be accessible with a mobile browser (for example, Chrome for Android, Firefox for Android, etc.). iOS devices will still be able to access Buzz through Safari.\\
			\\
			Buzz will be able to run on almost all operating systems that runs a modern web browser like Mozilla Firefox or Google Chrome (for example, Microsoft Windows, Apple Mac OSX or any distribution of Linux or BSD).
	\section{Architectural Patterns}
		For the design of Buzz, two patterns are considered: the \textbf{MVC (Model-View-Controller) pattern} and the \textbf{Layering architectural pattern}. For concurrency, the \textbf{Master-Slave pattern} is used.\\
		The top layers of the Model and the Controller are interfaces for the MVC architecture and the layers below provide the functionality.\\
		MVC is considered because:
		\begin{itemize}
			\item It provides modularity (i.e. the system's concerns are separated, thus easier to implement). (Solms, 2014)
			\item It allows for better maintainability (one can maintain the Model, View and Controller separately). (Solms, 2014)
			\item Testability (it is easier to test because of separated concerns, so the source of any problems are easy to identify). (Solms, 2014)
			\item Reuse (it is possible to take any component and reuse it where necessary). (Solms, 2014)
		\end{itemize} 
		Layering allows Buzz to have pluggable layers, which will allow the developers to replace layers as needed.
		This pattern allows for: 
		\begin{itemize}
			\item Improved cohesion. (Solms, 2014)
			\item Reduced complexity of the system. (Solms, 2014)
			\item Improved testability (which will allow for easier debugging). (Solms, 2014)
			\item Improved reuse and maintainability of the source code, because all the layers are individual and separate from one another. (Solms, 2014)
		\end{itemize}
		However, it should be mentioned that Layering has a performance overhead associated with it, as well as higher maintenance costs associated with the lower layers, because they impact the higher levels. (Solms, 2014). Given the benefits, however, the authors feel the reduced performance and maintenance costs is a good compromise for reduced complexity and testability.\\
		\\
		For concurrency, the Master-Slave architectural pattern (Solms, 2014) is considered, because the system needs to accommodate a large number of users at a time.
	\newpage
	\section{Bibliography}
		\begin{itemize}
			\item \textit{Basic MVC Architecture}. [Online]. Available: <http://www.tutorialspoint.com/struts\_2/basic\\\_mvc\_architecture.htm> [Accessed 5 March 2015].
			
			\item Beal, V. n.d. \textit{Linux Server}. [Online]. Available: <http://www.webopedia.com/TERM/L/linux\_\\server.html> [Accessed 10 March 2015].

			\item Bien, A. 2009. \textit{9 Reasons Why Java EE 6 Will Save Real Money - Or How To Convince Your Management}. [Online]. Available: <http://www.adam-bien.com/roller/abien/entry/\\8\_reasons\_why\_java\_ee> [Accessed 10 March 2015].

			\item Borges, B. 2013. \textit{Reasons to why I'm reconsidering JSF}. [Online]. Available: <http://blog.brunoborges.com.br/2013/01/reasons-to-why-im-reconsidering-jsf.html> [Accessed 10 March 2015].
			
			\item Davids, N.\textit{The Limitations of the Ethernet CRC and TCP/IP checksums for error detection} [Online]. Available: 
			<http://noahdavids.org/self-published/CRC-and-checksum.html>[Accessed 8 March 2015].

			\item \textit{Git --loval-branding-on-the-cheap}. [Online]. Available: <http://git-scm.com/> [Accessed 10 March 2015].

			\item \textit{JavaServer Faces Technology}. [Online]. Available: <http://www.oracle.com/technetwork/java/javaee/javaserverfaces-139869.html> [Accessed 10 March 2015].

			\item Kabanov, J. 2011. \textit{Ed Burns on Why JSF is the Most Popular Framework}. [Online]. Available: <http://zeroturnaround.com/rebellabs/ed-burns-on-why-jsf-is-the-most-popular-framework/> [Accessed 10 March 2015].
			\item Meyers, J. 1999. \textit{SMTP Service Extension for Authentication} [Online]. Available: <http://tools.ietf.org/html/rfc2554> [Accessed 9 March 2015]

			\item \textit{Securing FTP using SSH} [Online]. Available: <http://nurdletech.com/linux-notes/ftp/ssh.html> [Accessed 9 March 2015]

			\item \textit{PostgreSQL: About}. [Online]. Available: <http://www.postgresql.org/about/> [Accessed 10 March 2015].
			
			\item \textit{Software Engineering}. [Online]. Available: <http://sesolution.blogspot.com/p/software-engineering-layered-technology.html>[Accessed 5 March 2015].
								
			\item Solms, F. 2014. \textit{Software Architecture Desgin.} [Online]. University of Pretoria: Pretoria. Available: <http://www.cs.up.ac.za/modules/courses/download.php?id=8565> [Accessed 9 March 2015].
			
			\item Solms, F. 2015. \textit{Java Persistence API (JPA).} [Online]. University of Pretoria: Pretoria. Available: <http://www.cs.up.ac.za/modules/courses/download.php?id=8640> [Accessed 7 March 2015].
			
			\item \textit{SQL Server 2008 R2 Express Database Size Limit Increased to 10GB}. [Online]. Available: <http://blogs.msdn.com/b/sqlexpress/archive/2010/04/21/database-size-limit-increased-to-10gb-in-sql-server-2008-r2-express.aspx> [Accessed 9 March 2015].
			
			\item \textit{Usability Evaluation Basics}. [Online]. Available: <http://www.usability.gov/what-and-why/usability-evaluation.html> [Accessed 5 March 2015].

			\item W�hner, K. n.d. \textit{Why I will use Java EE (JEE, and not J2EE) instead of Spring in new Enterprise Java Projects in 2012}. [Online]. Available: <www.kai-waehner.de/blog/2011/11/21/why-i-will-use-java-ee-jee-and-not-j2ee-instead-of-spring-in-new-enterprise-java-projects-in-2012/> [Accessed 10 March 2015].		

			\item \textit{Welcome to Apache Maven}. [Online]. Available: <http://maven.apache.org/> [Accessed 10 March 2015].
			

			

			

			

			

			

		\end{itemize}
\bibliography{myrefs}{} 
\bibliographystyle{ieeetr}
\end{document}
\normalsize

\renewcommand{\thesection}{\arabic{section}}
\newpage
\begin{center}
\textsc{\LARGE Software Requirements Specification and Technology Neutral Process Design}\\[1.5cm]
\textsc{\Large Buzz Space Discussions/Mini Project}\\[0.5cm]
Further references see \href{https://github.com/ACalitz/COS301Phase2Group1A.git}{gitHub}.
\today
\end{center}
\tableofcontents{}
\newpage
\section{Preface}
		\subsection{Group Members}
			\begin{itemize}
				\item 14004489, Una Rambani�jo
				\item 13044924, Kevin Heritage
				\item 14211582, Tshepo Malesela
				\item 14035449, Vukile Langa
				\item 13230795, Wynand Meiring
				\item 14414555, Nontokozo Hlatshwayo
				\item 11152402, Tim Kirker
				\item 13057937, Thabang Letageng
				\item 13014171, Antonia Michael
			\end{itemize}
		\subsection{GitHub Repository}
			\url{https://github.com/ACalitz/COS301Phase2Group1A.git}
		\subsection{Contributions}
			\subsubsection{Access Channel Requirements}
				\begin{itemize}
					\item Everyone will work on these constraints
				\end{itemize}
			\subsubsection{Quality Requirements}  
				\begin{itemize}
					\item Rendani Dau
					\item Byron Dinkelmann
					\item Antonia Michael
				\end{itemize}
			\subsubsection{Integration Requirements}  
				\begin{itemize}
					\item Izak Blom
					\item Andre Calitz
					\item Chris Cloete
				\end{itemize}
			\subsubsection{Architecture Contraints}
				\begin{itemize}
					\item Daniel Christopher Alves Ara�jo
					\item Tim Kirker
					\item Thabang Letageng
				\end{itemize}
\index{Vision}
\newpage
	\section{Architectural Responsibilities}
	\begin{itemize}
	\item The system must be able to store threads in each buzz space, as well as provide for the creating, updating, deleting and viewing of all threads.
	\item The system must allow multiple users to access the buzz spaces, hence by means of the concurrency the system makes use of.
	\item The system must be scalable, auditable, usable, and reliable.
	\item The system must provide for notification messages to be sent to users if higher level users edit or delete their threads.
	\item The system should provide for multiple deployment.
	\item The system should allow users to up vote and down vote other user's posts as well as comment on them.
	\end{itemize}
\newpage
	\section{Quality Requirements}
	\subsection{Reliability}
	\subsubsection{Reasons for quality requirement}
	\begin{itemize}
	\item This has been given highest priority because Buzz spaces play an important role as users/students will receive important information from buzz spaces and they may be assessed for marks from their contributions to threads on certain buzz spaces. 
	\item To give all students a fair chance and to allow staff (lecturers, Teaching Assistants) to complete their duties (assessing the students) timeously, reliability plays a key role, ensuring that all functions work as the user expects them, when the user requires to use the system.
	\item It must hence have a maximum of an hour down time a day.
	\item Reliability stems further. Being defined as �the ability to be relied upon for accuracy�, one can take into account that the posts on the buzz space have been checked for plagiarism, similarity, netiquette and type of content based on the current status of the user. This hence provides reliability of the system. 
	\item The system is also expected to be reliable in terms of ensuring that users only get the privileges they are entitled to due to their particular status. Hence the system must also reflect an honest account of what rank each student has. 
	\end{itemize}
	\subsubsection{Strategies to achieve this quality requirement}
	\begin{itemize}
	\item Firstly, the prevention of faults. This is done by testing the system thoroughly, using resource locking as well as removing single points of failure. (Solms, 2014)
 \item Secondly, detection of faults, which is achieved through deadlock detection, logging, checkpoint evaluation and error communication to name but a few. Recovering from faults also falls under reliability. This is done by passive redundancy, maintaining backups and checkpoint rollbacks. (Solms, 2014)
 \end{itemize}
 \subsubsection{Patterns to achieve these strategies}
 \begin{itemize}
 \item MVC
 \end{itemize}
 The MVC pattern can  be used for reliability, because since the different layers are clearly separated, hence particular teams are focused on working on each layer, making the system more reliable. 
 \subsection{Auditability/Monitorability}
 \subsubsection{Reasons for quality requirements}
 \begin{itemize}
 \item Given the large user base the system will have, it is paramount that user activity should be track-able and changes made to buzz spaces such as creation of threads, deletion of threads, media uploads, etc. should be traceable back to the person who made these changes.
 \item Also, events that precede system crashes and those of unauthorized users should be traceable.
 \item In the event of a system crash, it should be possible to roll the system back to a previous working state
 \item This also involves high level users such as administrators and lectures being able to follow how
a student/user has participated in discussions, answering questions, asking questions and 
following lecturer input, and also being able to follow statistics such as how many people are active on the system.
 \item This falls under the monitorability aspect of the system. This is a
core quality requirement as students using the system get can get graded on their participation
as discussed above and is therefore a major requirement for the buzz system itself.
 \end{itemize}
 \subsubsection{Strategies to achieve this quality requirement}
 \begin{itemize}
 \item System should have log files running at all times to track all transactions made by users. 
 \item Time stamps should be added to document time and date information of the activities done so that the system can trace through the
information when needed, such as the events that precede a system crash or unauthorized access
that alters the system in any way.
\item System backup should allow rollback when needed.
\item ACID test can be carried out. Acid is an acronym that describes the properties of a database or system. The properties are:
	\begin{itemize}
	\item \textbf{Atomicity:} Defined as all or none situation referring to the processes that take place on the 
	   system. If something where to go wrong with a process such as posting on the system,
	   then the entire process has to be repeated or not at all.
	\item \textbf{Consistency:} All processes must be completed. No process can be left in a half-finished state,
	     if a failure is detected in a process then the entire process has to be rolled back.
	\item \textbf{Isolation:} Keeps process/transactions separate from one another until they are finished.
	\item \textbf{Durability:} The system must keep a backup of its current state so as to roll back to it if
	    the system where to experience a system failure, crash or corruption of data due
	    to a security breach.
	\end{itemize}
\item To ensure monitorability, post metadata, to document user involvement, should be saved. Also, post ranking system to document user quality in discussions.
 \end{itemize}
 \subsubsection{Patterns to achieve these strategies}
 \begin{itemize}
 \item MVC
 \item Layering
\end{itemize} 
 MVC is a suitable pattern because it provides auditability through logging all filter inputs and outputs (off queues). Layering is a suitable pattern because each separate layer can be audited and monitored individually, rather than auditing the system as a whole.
 \subsection{Usability}
 \subsubsection{Reasons for quality requirements}
 \begin{itemize}
 \item Since Buzz spaces may be a source of important information and discussion regarding academics (assignments, practicals etc.), it shouldn't be hard for new users to become familiarized with the system
	\item Especially because the initial user of the buzz space will indeed be a first year student who has probably had minimal exposure to discussion boards and how to use them. The system should also be rememberable hence it must also be understandable. 
	\item It should take at most 3 hours for an average user to be familiar with the system
 \end{itemize}
 \subsubsection{Strategies to achieve this quality requirement}
 \begin{itemize}
 \item Various goals of usability requirements are firstly, that the that the interface is intuitive, i.e. easy to navigate and understand, that the buttons and icons are self explanatory for the primary users.
 \item The interface must also not be a cluttered, frustrating and overwhelming one. 
 \item Ease of learning is also an important goal here such that users who have never used such a system can catch on easily and such that users who regularly use other discussion boards will not get confused and displaced. 
 \item The system must also be task efficient, i.e. if users access this space regularly, long tedious processes and other admin must be avoided.
\item Also, the colour schemes, functionality and interactiveness of the interface and system must contribute to this task efficiency. 
\item Different usability tests can be conducted such as handing out paper prototypes of different interface designs, and questionnaires getting feedback from the sample of people that were consulted in the survey. Problems with the different interfaces can be picked up during the usability testing phase, as indicated by the sample of users consulted, such that the final product will be much more user friendly. (http://www.usability.gov/what-and-why/usability-evaluation.html)
 \end{itemize}
 \subsubsection{Patterns to achieve these strategies}
 \begin{itemize}
 \item MVC
 \item Layering
\end{itemize}
MVC is a suitable pattern because the user will only need to interact with the front end interface, rather than dealing with the technical aspects of the back-end system. Another reason for this is that the developers allocated to working on the View will have the sole focus of making it usable.
Layering can be used within the subsections of MVC, i.e. the Control and Model layer can be layered to further divide concerns and allow different people to work on those layers.

\subsection{Scalability}
	\subsubsection{Reasons for quality requirement}
	\begin{itemize}
	\item This is a core requirement mainly because of the volume of students that will be accessing this discussion board i.e. all third/fourth year undergraduate students.
	\item Each of these students have approximately four COS modules in each year, hence this system will need to cater to this large mass of students as well as the lecturers, tutors and admin staff.
	\item With this, we can assume that there will be in excess of 2000 users meaning that the system has to have the ability to handle at least 1000 concurrent users at peak times.
	\item Different Buzz Spaces can have any number of threads which in turn may have resources (BuzzResources) such as media (videos, pod-casts and images) and documents (pdf, odf, doc, etc.) associated with them. This means that the system has to be able to manage the expansion of storage resources (mainly HDD's on the server)
	\end{itemize}
	\subsubsection{Strategies to achieve this quality requirement}
	\begin{itemize}
		\item We will need to firstly ensure that existing resources are managed efficiently, i.e. reducing the load using efficient storage, processing,  and persistence. In addition, we will need to ensure that the load is spread across resources and time, using methods of load balancing to spread load across resources as well as using scheduling and queueing to spread load across time.
		\item Secondly, the resources can be scaled up by increasing storage, increasing processing power and increasing the capacity of communication channels.
		\item Lastly, resources can be scaled out by means of using external resources, using commoditized resources and distributing tasks across specialized resources.
		\end{itemize}
	\subsubsection{Patterns to achieve these strategies}
	 \begin{itemize}
		\item Concurrency Master-Slave 
	\end{itemize}
We chose this pattern here due to the concurrency of the system, meaning that a large number of users must be able to access the system at a time.
 
 \subsection{Integrability}
 \subsubsection{Reasons for quality requirements}
 \begin{itemize}
 \item The buzz system is not a stand alone system as it requires an external database and website to integrate into. For example the LDAP Database, and Computer Science website.
\item The system must be portable to other system platforms so that it can be adapted to other client specific applications. For example another universities database and website.
\item It should take at most 1 day to integrate the system into another system.  
 \end{itemize}
 \subsubsection{Strategies to achieve this quality requirement}
 \begin{itemize}
 \item The interface must be structured between the system and database so that queries can be handled by any database with minimal changes to the interface to do so.
 \item The interface must be structured to be independent of external HTML and scripting languages so that it can be seamlessly integrated into any website or with minimal changes to either the system or website. 
 \end{itemize}
 \subsubsection{Patterns to achieve these strategies}
 \begin{itemize}
 \item MVC
 \end{itemize}
MVC is a sufficient pattern to use here, because firstly it is simplifies the systems through separation of concerns. Also, because it improves on reuse, the system can be integrated into other systems without having to create entirely new components.

\subsection{Nice to have}
\begin{itemize}
	\item Maintainability
	\item Flexibility
	\item Performance
	\item Security
\end{itemize}

\newpage
	\section{Integration Requirements} 
	\subsection{Integration channels}
	\subsubsection{LDAP Database System}
	LDAP, the database used by the computer science department. Integration with this system is required for retrieval of the majority of the information that the buzz system will require. Performance for this channel is also critical and thus we 				recommend the ftp protocol as it will allow quick access to the system so that all the 			required information can be obtained as fast as possible. The reduced security of this protocol is negligible as the connections will all be local on the same server and OpenSSH can be used for additional security (See FTP protocol discussion below.
	\paragraph{Quality Requirements for the Database System}
	\begin{itemize}
	\item{\textbf{Performance:} All queries to the LDAP database should be rapid and efficient to provide the best user experience to the maximum amount of people. This can be achieved through interacting with the system using the ftp protocol since the Buzz system will be hosted on the same servers as the LDAP system.}
	\item{\textbf{Reliability:} This is also an important requirement as the Buzz system needs to have as little downtime as possible and thus a reliable connection with the LDAP server is required. }
	\item{\textbf{Scalability:} The system needs to be able to work with a large amount of users concurrently.}
	\item{\textbf{Integrability:} The integration with the LDAP database should be engineered in a manner which can easily be adjusted to accommodate additional integrations as the LDAP system evolves.}
	\item{\textbf{Affordability:} Access or queries made to the LDAP database should be affordable. That is, it should be possible to make requests as often as needed without any implications on the overall system performance.}
	\item{\textbf{Flexibility:} The integration with the LDAP database should be flexible in the sense that slight changes in the database functionality should not affect the integration with the Buzz System.}
	\item{\textbf{Auditability:} All information changes, user interactions, database queries should be auditable. i.e. Identification, time-stamps etcetera should be linked to every action with regards to the integration channels to be used. Every integration action must be traceable to a user or system action.}
	\end{itemize}

  \subsubsection{Web services}
  Web services namely HTTP and TCP, will facilitate user interaction with the Buzz system. 
	\paragraph{Quality Requirements for the Web Services}
	\begin{itemize}
	\item{\textbf{Security:} This is an important quality requirement in any web-based application that incorporates authentication and the storage of personal information.}
	\item{\textbf{Reliability:} This is also an important requirement as the Buzz system needs to have as little downtime as possible and Web service integrations should be functional at all times. }
	\item{\textbf{Integrability:} Is of importance as this service needs to integrate with a host of other services that provide some other form of functionality to the buzz system, be it the notification system, authentication system etc.}
	\item{\textbf{Performance:} The performance of these integration channels should be a priority to avoid lowering the overall Buzz system performance. This negatively implicates the use of https which performs somewhat slower than http, but the increased security provides enough motivation to make use of https.}
	\item{\textbf{Scalability:} The protocols used need to be competent to address at least 10 000 requests per second for example to ensure optimal system performance and prevent breakdown.}
	\end{itemize}
\subsection{Access channels}
	Buzz has various access channels from which users can gain access to the system:
	\begin{itemize}
		\item All users of Buzz (mostly students, but will still be accessed by lecturers, administrators, etc.) will have access to the system via the Computer Science website. Users will use a modern web browser, such as Mozilla Firefox, Google Chrome, Microsoft Internet Explorer, or Apple Safari to interact with the system. Students and lecturers (or administrators) will, however, have different permissions. Thus, the interface will be altered depending on the permissions the user has.
		\item Users will also have access to Buzz via an Android application.
	\end{itemize} 	


\subsection{Protocols}
\subsubsection{Database Query Language}
	SQL will be used to query the LDAP database by making use of the PostgreSQL database management system. \textbf{This is not classified as a protocol, but is mentioned here for the sake of clarity.}
\subsubsection{HTTP - Hypertext Transfer Protocol}
Integration with this protocol will occur at a high level and typically be handled by libraries or browser-clients etc.
\textbf{To be used for:	}
	\begin{itemize}
	\item{All data transferred between users and the server on which the system is hosted.}
	\item{All data transferred between the system and LDAP.}
	\item{Transfer of miscellaneous data such as HTTP error codes to ensure both servers and clients are aware of the state of data transfers and its results}
	\end{itemize}
\subsubsection{TCP - Transmission Control Protocol}
 For establishing network connections between the user computers and the system server as well as between the system server and the LDAP server. Streams of data can then be exchanged between the connected hosts. Error detection, faulty transmission of data, resending of data etc. will all be done using TCP (Davids). Integration with this protocol will occur at a high level and will typically be handled by libraries or operating system functions.
\subsubsection{FTP - File Transfer Protocol}
To enhance system performance, FTP will be used for data transfer between the LDAP database and the Buzz system wherever possible, since both systems will possibly be hosted on the same server and therefore the lowered security (nurdletech.com) of the FTP protocol is negligible to some extent. OpenSSH (Open Secure Shell) can alternatively used to secure FTP connections.
\subsubsection{SMTP - Simple Mail Transfer Protocol}
This protocol will be used to handle e-mail communication related to the Buzz system. It addresses Security as a quality requirement since it incorporates SMTP-Authentication defined by RFC 2554 (Meyers, 1999) which enhances the security of the protocol.

\subsection{API specifications}
\subsubsection{Apache Maven}
Apache Maven this can be used to help with the build process of the system as it will allow for easy integration with other external services.
\subsubsection{WSDL - Web Services Description Language}
WSDL can be used for information exchange between systems as it provides an effective means of sending messages over the network.
\subsubsection{JPA - Java Persistence API}
JPA can be used if a java database is used.
\subsubsection{PostgreSQL}
PostgreSQL can be used as an alternative means for interacting with a database in the case that a java based database is not used.
\subsubsection{GIT}
GIT can be used as a version control API as it will allow an easy means to manage the versions and also solve code conflicts.


\newpage
	\section{Architecture Constraints}
		\subsection{Reference Architecture}
			Buzz's system architecture will be using Java EE as a reference architecture, as specified by the client. Reasons for the use of Java EE is:
			\begin{itemize}
				\item It is a set of standard specifications and is  therefore independent of any particular vendor. Often, there are a number of implementations of the Java EE specifications. (W�hner, n.d.).
				\item Sustainability. (W�hner, n.d.).
				\item Portability (projects are easy to migrate to Java EE). (Bien, 2009).
				\item Java EE implementations are lightweight. (Bien, 2009) (W�hner, n.d.).
			\end{itemize}
		\subsection{Technologies}
			The primary programming language used for Buzz is \textbf{Java}.
			A few additional technologies will be used in the development of Buzz:
			\begin{itemize}
				\item \textbf{Apache Maven}: Maven is a project management and build tool for Java. (http://maven.apache.org/).
				\item \textbf{Git and GitHub}: A version control software and a repository website that will be used to host the source code of Buzz. Reasons for the use of Git is its ease of use and because it is free and open source. (http://git-scm.com/).
				\item \textbf{JPA (Java Persistence API)}: It is an API for Java that will describe the management of relational data. (Solms, 2015).
				\item \textbf{JSF (JavaServer Faces)}: It is a specification for building server-based user interfaces. (http://www.oracle.com).\\ Several reasons for using JSF, including the ability to define a page using HTML and the ease of composing custom, reusable components. (Borges, 2013).
				\item \textbf{PostgreSQL}: PostgreSQL is a free, open-source, cross-platform, object-orientated database management system. (http://www.postgresql.org/about/).\\
				The reason we chose it is because of its unlimited maximum database size, its compliance with ANSI-SQL and features like Multi-Version Concurrency Control. (http://www.postgresql.org/about/).\\
				The reason for using PostgreSQL rather than Microsoft SQL Server Express is because it has restrictions, such as a maximum database size of 10GB (http://blogs.msdn.com), which the authors believe may be too limited.
				\item \textbf{JPQL (Java Persistence Query Language)}: JPQL is a technology-neutral object-orientated query language used to "formulate queries across object graphs." (Solms, 2015)
			\end{itemize}
		\subsection{Operating Systems}
			Buzz will be designed to run on a Linux-based operating system. Linux-based operating systems are free, open-source and provide a stable base for the system. (Beal, n.d.).\\
			\\
			A Google Android client will be designed for Buzz that will allow users to log into and interact with Buzz on their phones. However, considering that Buzz will be linked to the Computer Science website, it will still be accessible with a mobile browser (for example, Chrome for Android, Firefox for Android, etc.). iOS devices will still be able to access Buzz through Safari.\\
			\\
			Buzz will be able to run on almost all operating systems that runs a modern web browser like Mozilla Firefox or Google Chrome (for example, Microsoft Windows, Apple Mac OSX or any distribution of Linux or BSD).
	\section{Architectural Patterns}
		For the design of Buzz, two patterns are considered: the \textbf{MVC (Model-View-Controller) pattern} and the \textbf{Layering architectural pattern}. For concurrency, the \textbf{Master-Slave pattern} is used.\\
		The top layers of the Model and the Controller are interfaces for the MVC architecture and the layers below provide the functionality.\\
		MVC is considered because:
		\begin{itemize}
			\item It provides modularity (i.e. the system's concerns are separated, thus easier to implement). (Solms, 2014)
			\item It allows for better maintainability (one can maintain the Model, View and Controller separately). (Solms, 2014)
			\item Testability (it is easier to test because of separated concerns, so the source of any problems are easy to identify). (Solms, 2014)
			\item Reuse (it is possible to take any component and reuse it where necessary). (Solms, 2014)
		\end{itemize} 
		Layering allows Buzz to have pluggable layers, which will allow the developers to replace layers as needed.
		This pattern allows for: 
		\begin{itemize}
			\item Improved cohesion. (Solms, 2014)
			\item Reduced complexity of the system. (Solms, 2014)
			\item Improved testability (which will allow for easier debugging). (Solms, 2014)
			\item Improved reuse and maintainability of the source code, because all the layers are individual and separate from one another. (Solms, 2014)
		\end{itemize}
		However, it should be mentioned that Layering has a performance overhead associated with it, as well as higher maintenance costs associated with the lower layers, because they impact the higher levels. (Solms, 2014). Given the benefits, however, the authors feel the reduced performance and maintenance costs is a good compromise for reduced complexity and testability.\\
		\\
		For concurrency, the Master-Slave architectural pattern (Solms, 2014) is considered, because the system needs to accommodate a large number of users at a time.
	\newpage
	\section{Bibliography}
		\begin{itemize}
			\item \textit{Basic MVC Architecture}. [Online]. Available: <http://www.tutorialspoint.com/struts\_2/basic\\\_mvc\_architecture.htm> [Accessed 5 March 2015].
			
			\item Beal, V. n.d. \textit{Linux Server}. [Online]. Available: <http://www.webopedia.com/TERM/L/linux\_\\server.html> [Accessed 10 March 2015].

			\item Bien, A. 2009. \textit{9 Reasons Why Java EE 6 Will Save Real Money - Or How To Convince Your Management}. [Online]. Available: <http://www.adam-bien.com/roller/abien/entry/\\8\_reasons\_why\_java\_ee> [Accessed 10 March 2015].

			\item Borges, B. 2013. \textit{Reasons to why I'm reconsidering JSF}. [Online]. Available: <http://blog.brunoborges.com.br/2013/01/reasons-to-why-im-reconsidering-jsf.html> [Accessed 10 March 2015].
			
			\item Davids, N.\textit{The Limitations of the Ethernet CRC and TCP/IP checksums for error detection} [Online]. Available: 
			<http://noahdavids.org/self-published/CRC-and-checksum.html>[Accessed 8 March 2015].

			\item \textit{Git --loval-branding-on-the-cheap}. [Online]. Available: <http://git-scm.com/> [Accessed 10 March 2015].

			\item \textit{JavaServer Faces Technology}. [Online]. Available: <http://www.oracle.com/technetwork/java/javaee/javaserverfaces-139869.html> [Accessed 10 March 2015].

			\item Kabanov, J. 2011. \textit{Ed Burns on Why JSF is the Most Popular Framework}. [Online]. Available: <http://zeroturnaround.com/rebellabs/ed-burns-on-why-jsf-is-the-most-popular-framework/> [Accessed 10 March 2015].
			\item Meyers, J. 1999. \textit{SMTP Service Extension for Authentication} [Online]. Available: <http://tools.ietf.org/html/rfc2554> [Accessed 9 March 2015]

			\item \textit{Securing FTP using SSH} [Online]. Available: <http://nurdletech.com/linux-notes/ftp/ssh.html> [Accessed 9 March 2015]

			\item \textit{PostgreSQL: About}. [Online]. Available: <http://www.postgresql.org/about/> [Accessed 10 March 2015].
			
			\item \textit{Software Engineering}. [Online]. Available: <http://sesolution.blogspot.com/p/software-engineering-layered-technology.html>[Accessed 5 March 2015].
								
			\item Solms, F. 2014. \textit{Software Architecture Desgin.} [Online]. University of Pretoria: Pretoria. Available: <http://www.cs.up.ac.za/modules/courses/download.php?id=8565> [Accessed 9 March 2015].
			
			\item Solms, F. 2015. \textit{Java Persistence API (JPA).} [Online]. University of Pretoria: Pretoria. Available: <http://www.cs.up.ac.za/modules/courses/download.php?id=8640> [Accessed 7 March 2015].
			
			\item \textit{SQL Server 2008 R2 Express Database Size Limit Increased to 10GB}. [Online]. Available: <http://blogs.msdn.com/b/sqlexpress/archive/2010/04/21/database-size-limit-increased-to-10gb-in-sql-server-2008-r2-express.aspx> [Accessed 9 March 2015].
			
			\item \textit{Usability Evaluation Basics}. [Online]. Available: <http://www.usability.gov/what-and-why/usability-evaluation.html> [Accessed 5 March 2015].

			\item W�hner, K. n.d. \textit{Why I will use Java EE (JEE, and not J2EE) instead of Spring in new Enterprise Java Projects in 2012}. [Online]. Available: <www.kai-waehner.de/blog/2011/11/21/why-i-will-use-java-ee-jee-and-not-j2ee-instead-of-spring-in-new-enterprise-java-projects-in-2012/> [Accessed 10 March 2015].		

			\item \textit{Welcome to Apache Maven}. [Online]. Available: <http://maven.apache.org/> [Accessed 10 March 2015].
			

			

			

			

			

			

		\end{itemize}
\bibliography{myrefs}{} 
\bibliographystyle{ieeetr}
\end{document}
\normalsize

\renewcommand{\thesection}{\arabic{section}}
\newpage
\begin{center}
\textsc{\LARGE Software Requirements Specification and Technology Neutral Process Design}\\[1.5cm]
\textsc{\Large Buzz Space Discussions/Mini Project}\\[0.5cm]
Further references see \href{https://github.com/ACalitz/COS301Phase2Group1A.git}{gitHub}.
\today
\end{center}
\tableofcontents{}
\newpage
\section{Preface}
		\subsection{Group Members}
			\begin{itemize}
				\item 14004489, Una Rambani�jo
				\item 13044924, Kevin Heritage
				\item 14211582, Tshepo Malesela
				\item 14035449, Vukile Langa
				\item 13230795, Wynand Meiring
				\item 14414555, Nontokozo Hlatshwayo
				\item 11152402, Tim Kirker
				\item 13057937, Thabang Letageng
				\item 13014171, Antonia Michael
			\end{itemize}
		\subsection{GitHub Repository}
			\url{https://github.com/ACalitz/COS301Phase2Group1A.git}
		\subsection{Contributions}
			\subsubsection{Access Channel Requirements}
				\begin{itemize}
					\item Everyone will work on these constraints
				\end{itemize}
			\subsubsection{Quality Requirements}  
				\begin{itemize}
					\item Rendani Dau
					\item Byron Dinkelmann
					\item Antonia Michael
				\end{itemize}
			\subsubsection{Integration Requirements}  
				\begin{itemize}
					\item Izak Blom
					\item Andre Calitz
					\item Chris Cloete
				\end{itemize}
			\subsubsection{Architecture Contraints}
				\begin{itemize}
					\item Daniel Christopher Alves Ara�jo
					\item Tim Kirker
					\item Thabang Letageng
				\end{itemize}
\index{Vision}
\newpage
	\section{Architectural Responsibilities}
	\begin{itemize}
	\item The system must be able to store threads in each buzz space, as well as provide for the creating, updating, deleting and viewing of all threads.
	\item The system must allow multiple users to access the buzz spaces, hence by means of the concurrency the system makes use of.
	\item The system must be scalable, auditable, usable, and reliable.
	\item The system must provide for notification messages to be sent to users if higher level users edit or delete their threads.
	\item The system should provide for multiple deployment.
	\item The system should allow users to up vote and down vote other user's posts as well as comment on them.
	\end{itemize}
\newpage
	\section{Quality Requirements}
	\subsection{Reliability}
	\subsubsection{Reasons for quality requirement}
	\begin{itemize}
	\item This has been given highest priority because Buzz spaces play an important role as users/students will receive important information from buzz spaces and they may be assessed for marks from their contributions to threads on certain buzz spaces. 
	\item To give all students a fair chance and to allow staff (lecturers, Teaching Assistants) to complete their duties (assessing the students) timeously, reliability plays a key role, ensuring that all functions work as the user expects them, when the user requires to use the system.
	\item It must hence have a maximum of an hour down time a day.
	\item Reliability stems further. Being defined as �the ability to be relied upon for accuracy�, one can take into account that the posts on the buzz space have been checked for plagiarism, similarity, netiquette and type of content based on the current status of the user. This hence provides reliability of the system. 
	\item The system is also expected to be reliable in terms of ensuring that users only get the privileges they are entitled to due to their particular status. Hence the system must also reflect an honest account of what rank each student has. 
	\end{itemize}
	\subsubsection{Strategies to achieve this quality requirement}
	\begin{itemize}
	\item Firstly, the prevention of faults. This is done by testing the system thoroughly, using resource locking as well as removing single points of failure. (Solms, 2014)
 \item Secondly, detection of faults, which is achieved through deadlock detection, logging, checkpoint evaluation and error communication to name but a few. Recovering from faults also falls under reliability. This is done by passive redundancy, maintaining backups and checkpoint rollbacks. (Solms, 2014)
 \end{itemize}
 \subsubsection{Patterns to achieve these strategies}
 \begin{itemize}
 \item MVC
 \end{itemize}
 The MVC pattern can  be used for reliability, because since the different layers are clearly separated, hence particular teams are focused on working on each layer, making the system more reliable. 
 \subsection{Auditability/Monitorability}
 \subsubsection{Reasons for quality requirements}
 \begin{itemize}
 \item Given the large user base the system will have, it is paramount that user activity should be track-able and changes made to buzz spaces such as creation of threads, deletion of threads, media uploads, etc. should be traceable back to the person who made these changes.
 \item Also, events that precede system crashes and those of unauthorized users should be traceable.
 \item In the event of a system crash, it should be possible to roll the system back to a previous working state
 \item This also involves high level users such as administrators and lectures being able to follow how
a student/user has participated in discussions, answering questions, asking questions and 
following lecturer input, and also being able to follow statistics such as how many people are active on the system.
 \item This falls under the monitorability aspect of the system. This is a
core quality requirement as students using the system get can get graded on their participation
as discussed above and is therefore a major requirement for the buzz system itself.
 \end{itemize}
 \subsubsection{Strategies to achieve this quality requirement}
 \begin{itemize}
 \item System should have log files running at all times to track all transactions made by users. 
 \item Time stamps should be added to document time and date information of the activities done so that the system can trace through the
information when needed, such as the events that precede a system crash or unauthorized access
that alters the system in any way.
\item System backup should allow rollback when needed.
\item ACID test can be carried out. Acid is an acronym that describes the properties of a database or system. The properties are:
	\begin{itemize}
	\item \textbf{Atomicity:} Defined as all or none situation referring to the processes that take place on the 
	   system. If something where to go wrong with a process such as posting on the system,
	   then the entire process has to be repeated or not at all.
	\item \textbf{Consistency:} All processes must be completed. No process can be left in a half-finished state,
	     if a failure is detected in a process then the entire process has to be rolled back.
	\item \textbf{Isolation:} Keeps process/transactions separate from one another until they are finished.
	\item \textbf{Durability:} The system must keep a backup of its current state so as to roll back to it if
	    the system where to experience a system failure, crash or corruption of data due
	    to a security breach.
	\end{itemize}
\item To ensure monitorability, post metadata, to document user involvement, should be saved. Also, post ranking system to document user quality in discussions.
 \end{itemize}
 \subsubsection{Patterns to achieve these strategies}
 \begin{itemize}
 \item MVC
 \item Layering
\end{itemize} 
 MVC is a suitable pattern because it provides auditability through logging all filter inputs and outputs (off queues). Layering is a suitable pattern because each separate layer can be audited and monitored individually, rather than auditing the system as a whole.
 \subsection{Usability}
 \subsubsection{Reasons for quality requirements}
 \begin{itemize}
 \item Since Buzz spaces may be a source of important information and discussion regarding academics (assignments, practicals etc.), it shouldn't be hard for new users to become familiarized with the system
	\item Especially because the initial user of the buzz space will indeed be a first year student who has probably had minimal exposure to discussion boards and how to use them. The system should also be rememberable hence it must also be understandable. 
	\item It should take at most 3 hours for an average user to be familiar with the system
 \end{itemize}
 \subsubsection{Strategies to achieve this quality requirement}
 \begin{itemize}
 \item Various goals of usability requirements are firstly, that the that the interface is intuitive, i.e. easy to navigate and understand, that the buttons and icons are self explanatory for the primary users.
 \item The interface must also not be a cluttered, frustrating and overwhelming one. 
 \item Ease of learning is also an important goal here such that users who have never used such a system can catch on easily and such that users who regularly use other discussion boards will not get confused and displaced. 
 \item The system must also be task efficient, i.e. if users access this space regularly, long tedious processes and other admin must be avoided.
\item Also, the colour schemes, functionality and interactiveness of the interface and system must contribute to this task efficiency. 
\item Different usability tests can be conducted such as handing out paper prototypes of different interface designs, and questionnaires getting feedback from the sample of people that were consulted in the survey. Problems with the different interfaces can be picked up during the usability testing phase, as indicated by the sample of users consulted, such that the final product will be much more user friendly. (http://www.usability.gov/what-and-why/usability-evaluation.html)
 \end{itemize}
 \subsubsection{Patterns to achieve these strategies}
 \begin{itemize}
 \item MVC
 \item Layering
\end{itemize}
MVC is a suitable pattern because the user will only need to interact with the front end interface, rather than dealing with the technical aspects of the back-end system. Another reason for this is that the developers allocated to working on the View will have the sole focus of making it usable.
Layering can be used within the subsections of MVC, i.e. the Control and Model layer can be layered to further divide concerns and allow different people to work on those layers.

\subsection{Scalability}
	\subsubsection{Reasons for quality requirement}
	\begin{itemize}
	\item This is a core requirement mainly because of the volume of students that will be accessing this discussion board i.e. all third/fourth year undergraduate students.
	\item Each of these students have approximately four COS modules in each year, hence this system will need to cater to this large mass of students as well as the lecturers, tutors and admin staff.
	\item With this, we can assume that there will be in excess of 2000 users meaning that the system has to have the ability to handle at least 1000 concurrent users at peak times.
	\item Different Buzz Spaces can have any number of threads which in turn may have resources (BuzzResources) such as media (videos, pod-casts and images) and documents (pdf, odf, doc, etc.) associated with them. This means that the system has to be able to manage the expansion of storage resources (mainly HDD's on the server)
	\end{itemize}
	\subsubsection{Strategies to achieve this quality requirement}
	\begin{itemize}
		\item We will need to firstly ensure that existing resources are managed efficiently, i.e. reducing the load using efficient storage, processing,  and persistence. In addition, we will need to ensure that the load is spread across resources and time, using methods of load balancing to spread load across resources as well as using scheduling and queueing to spread load across time.
		\item Secondly, the resources can be scaled up by increasing storage, increasing processing power and increasing the capacity of communication channels.
		\item Lastly, resources can be scaled out by means of using external resources, using commoditized resources and distributing tasks across specialized resources.
		\end{itemize}
	\subsubsection{Patterns to achieve these strategies}
	 \begin{itemize}
		\item Concurrency Master-Slave 
	\end{itemize}
We chose this pattern here due to the concurrency of the system, meaning that a large number of users must be able to access the system at a time.
 
 \subsection{Integrability}
 \subsubsection{Reasons for quality requirements}
 \begin{itemize}
 \item The buzz system is not a stand alone system as it requires an external database and website to integrate into. For example the LDAP Database, and Computer Science website.
\item The system must be portable to other system platforms so that it can be adapted to other client specific applications. For example another universities database and website.
\item It should take at most 1 day to integrate the system into another system.  
 \end{itemize}
 \subsubsection{Strategies to achieve this quality requirement}
 \begin{itemize}
 \item The interface must be structured between the system and database so that queries can be handled by any database with minimal changes to the interface to do so.
 \item The interface must be structured to be independent of external HTML and scripting languages so that it can be seamlessly integrated into any website or with minimal changes to either the system or website. 
 \end{itemize}
 \subsubsection{Patterns to achieve these strategies}
 \begin{itemize}
 \item MVC
 \end{itemize}
MVC is a sufficient pattern to use here, because firstly it is simplifies the systems through separation of concerns. Also, because it improves on reuse, the system can be integrated into other systems without having to create entirely new components.

\subsection{Nice to have}
\begin{itemize}
	\item Maintainability
	\item Flexibility
	\item Performance
	\item Security
\end{itemize}

\newpage
	\section{Integration Requirements} 
	\subsection{Integration channels}
	\subsubsection{LDAP Database System}
	LDAP, the database used by the computer science department. Integration with this system is required for retrieval of the majority of the information that the buzz system will require. Performance for this channel is also critical and thus we 				recommend the ftp protocol as it will allow quick access to the system so that all the 			required information can be obtained as fast as possible. The reduced security of this protocol is negligible as the connections will all be local on the same server and OpenSSH can be used for additional security (See FTP protocol discussion below.
	\paragraph{Quality Requirements for the Database System}
	\begin{itemize}
	\item{\textbf{Performance:} All queries to the LDAP database should be rapid and efficient to provide the best user experience to the maximum amount of people. This can be achieved through interacting with the system using the ftp protocol since the Buzz system will be hosted on the same servers as the LDAP system.}
	\item{\textbf{Reliability:} This is also an important requirement as the Buzz system needs to have as little downtime as possible and thus a reliable connection with the LDAP server is required. }
	\item{\textbf{Scalability:} The system needs to be able to work with a large amount of users concurrently.}
	\item{\textbf{Integrability:} The integration with the LDAP database should be engineered in a manner which can easily be adjusted to accommodate additional integrations as the LDAP system evolves.}
	\item{\textbf{Affordability:} Access or queries made to the LDAP database should be affordable. That is, it should be possible to make requests as often as needed without any implications on the overall system performance.}
	\item{\textbf{Flexibility:} The integration with the LDAP database should be flexible in the sense that slight changes in the database functionality should not affect the integration with the Buzz System.}
	\item{\textbf{Auditability:} All information changes, user interactions, database queries should be auditable. i.e. Identification, time-stamps etcetera should be linked to every action with regards to the integration channels to be used. Every integration action must be traceable to a user or system action.}
	\end{itemize}

  \subsubsection{Web services}
  Web services namely HTTP and TCP, will facilitate user interaction with the Buzz system. 
	\paragraph{Quality Requirements for the Web Services}
	\begin{itemize}
	\item{\textbf{Security:} This is an important quality requirement in any web-based application that incorporates authentication and the storage of personal information.}
	\item{\textbf{Reliability:} This is also an important requirement as the Buzz system needs to have as little downtime as possible and Web service integrations should be functional at all times. }
	\item{\textbf{Integrability:} Is of importance as this service needs to integrate with a host of other services that provide some other form of functionality to the buzz system, be it the notification system, authentication system etc.}
	\item{\textbf{Performance:} The performance of these integration channels should be a priority to avoid lowering the overall Buzz system performance. This negatively implicates the use of https which performs somewhat slower than http, but the increased security provides enough motivation to make use of https.}
	\item{\textbf{Scalability:} The protocols used need to be competent to address at least 10 000 requests per second for example to ensure optimal system performance and prevent breakdown.}
	\end{itemize}
\subsection{Access channels}
	Buzz has various access channels from which users can gain access to the system:
	\begin{itemize}
		\item All users of Buzz (mostly students, but will still be accessed by lecturers, administrators, etc.) will have access to the system via the Computer Science website. Users will use a modern web browser, such as Mozilla Firefox, Google Chrome, Microsoft Internet Explorer, or Apple Safari to interact with the system. Students and lecturers (or administrators) will, however, have different permissions. Thus, the interface will be altered depending on the permissions the user has.
		\item Users will also have access to Buzz via an Android application.
	\end{itemize} 	


\subsection{Protocols}
\subsubsection{Database Query Language}
	SQL will be used to query the LDAP database by making use of the PostgreSQL database management system. \textbf{This is not classified as a protocol, but is mentioned here for the sake of clarity.}
\subsubsection{HTTP - Hypertext Transfer Protocol}
Integration with this protocol will occur at a high level and typically be handled by libraries or browser-clients etc.
\textbf{To be used for:	}
	\begin{itemize}
	\item{All data transferred between users and the server on which the system is hosted.}
	\item{All data transferred between the system and LDAP.}
	\item{Transfer of miscellaneous data such as HTTP error codes to ensure both servers and clients are aware of the state of data transfers and its results}
	\end{itemize}
\subsubsection{TCP - Transmission Control Protocol}
 For establishing network connections between the user computers and the system server as well as between the system server and the LDAP server. Streams of data can then be exchanged between the connected hosts. Error detection, faulty transmission of data, resending of data etc. will all be done using TCP (Davids). Integration with this protocol will occur at a high level and will typically be handled by libraries or operating system functions.
\subsubsection{FTP - File Transfer Protocol}
To enhance system performance, FTP will be used for data transfer between the LDAP database and the Buzz system wherever possible, since both systems will possibly be hosted on the same server and therefore the lowered security (nurdletech.com) of the FTP protocol is negligible to some extent. OpenSSH (Open Secure Shell) can alternatively used to secure FTP connections.
\subsubsection{SMTP - Simple Mail Transfer Protocol}
This protocol will be used to handle e-mail communication related to the Buzz system. It addresses Security as a quality requirement since it incorporates SMTP-Authentication defined by RFC 2554 (Meyers, 1999) which enhances the security of the protocol.

\subsection{API specifications}
\subsubsection{Apache Maven}
Apache Maven this can be used to help with the build process of the system as it will allow for easy integration with other external services.
\subsubsection{WSDL - Web Services Description Language}
WSDL can be used for information exchange between systems as it provides an effective means of sending messages over the network.
\subsubsection{JPA - Java Persistence API}
JPA can be used if a java database is used.
\subsubsection{PostgreSQL}
PostgreSQL can be used as an alternative means for interacting with a database in the case that a java based database is not used.
\subsubsection{GIT}
GIT can be used as a version control API as it will allow an easy means to manage the versions and also solve code conflicts.


\newpage
	\section{Architecture Constraints}
		\subsection{Reference Architecture}
			Buzz's system architecture will be using Java EE as a reference architecture, as specified by the client. Reasons for the use of Java EE is:
			\begin{itemize}
				\item It is a set of standard specifications and is  therefore independent of any particular vendor. Often, there are a number of implementations of the Java EE specifications. (W�hner, n.d.).
				\item Sustainability. (W�hner, n.d.).
				\item Portability (projects are easy to migrate to Java EE). (Bien, 2009).
				\item Java EE implementations are lightweight. (Bien, 2009) (W�hner, n.d.).
			\end{itemize}
		\subsection{Technologies}
			The primary programming language used for Buzz is \textbf{Java}.
			A few additional technologies will be used in the development of Buzz:
			\begin{itemize}
				\item \textbf{Apache Maven}: Maven is a project management and build tool for Java. (http://maven.apache.org/).
				\item \textbf{Git and GitHub}: A version control software and a repository website that will be used to host the source code of Buzz. Reasons for the use of Git is its ease of use and because it is free and open source. (http://git-scm.com/).
				\item \textbf{JPA (Java Persistence API)}: It is an API for Java that will describe the management of relational data. (Solms, 2015).
				\item \textbf{JSF (JavaServer Faces)}: It is a specification for building server-based user interfaces. (http://www.oracle.com).\\ Several reasons for using JSF, including the ability to define a page using HTML and the ease of composing custom, reusable components. (Borges, 2013).
				\item \textbf{PostgreSQL}: PostgreSQL is a free, open-source, cross-platform, object-orientated database management system. (http://www.postgresql.org/about/).\\
				The reason we chose it is because of its unlimited maximum database size, its compliance with ANSI-SQL and features like Multi-Version Concurrency Control. (http://www.postgresql.org/about/).\\
				The reason for using PostgreSQL rather than Microsoft SQL Server Express is because it has restrictions, such as a maximum database size of 10GB (http://blogs.msdn.com), which the authors believe may be too limited.
				\item \textbf{JPQL (Java Persistence Query Language)}: JPQL is a technology-neutral object-orientated query language used to "formulate queries across object graphs." (Solms, 2015)
			\end{itemize}
		\subsection{Operating Systems}
			Buzz will be designed to run on a Linux-based operating system. Linux-based operating systems are free, open-source and provide a stable base for the system. (Beal, n.d.).\\
			\\
			A Google Android client will be designed for Buzz that will allow users to log into and interact with Buzz on their phones. However, considering that Buzz will be linked to the Computer Science website, it will still be accessible with a mobile browser (for example, Chrome for Android, Firefox for Android, etc.). iOS devices will still be able to access Buzz through Safari.\\
			\\
			Buzz will be able to run on almost all operating systems that runs a modern web browser like Mozilla Firefox or Google Chrome (for example, Microsoft Windows, Apple Mac OSX or any distribution of Linux or BSD).
	\section{Architectural Patterns}
		For the design of Buzz, two patterns are considered: the \textbf{MVC (Model-View-Controller) pattern} and the \textbf{Layering architectural pattern}. For concurrency, the \textbf{Master-Slave pattern} is used.\\
		The top layers of the Model and the Controller are interfaces for the MVC architecture and the layers below provide the functionality.\\
		MVC is considered because:
		\begin{itemize}
			\item It provides modularity (i.e. the system's concerns are separated, thus easier to implement). (Solms, 2014)
			\item It allows for better maintainability (one can maintain the Model, View and Controller separately). (Solms, 2014)
			\item Testability (it is easier to test because of separated concerns, so the source of any problems are easy to identify). (Solms, 2014)
			\item Reuse (it is possible to take any component and reuse it where necessary). (Solms, 2014)
		\end{itemize} 
		Layering allows Buzz to have pluggable layers, which will allow the developers to replace layers as needed.
		This pattern allows for: 
		\begin{itemize}
			\item Improved cohesion. (Solms, 2014)
			\item Reduced complexity of the system. (Solms, 2014)
			\item Improved testability (which will allow for easier debugging). (Solms, 2014)
			\item Improved reuse and maintainability of the source code, because all the layers are individual and separate from one another. (Solms, 2014)
		\end{itemize}
		However, it should be mentioned that Layering has a performance overhead associated with it, as well as higher maintenance costs associated with the lower layers, because they impact the higher levels. (Solms, 2014). Given the benefits, however, the authors feel the reduced performance and maintenance costs is a good compromise for reduced complexity and testability.\\
		\\
		For concurrency, the Master-Slave architectural pattern (Solms, 2014) is considered, because the system needs to accommodate a large number of users at a time.
	\newpage
	\section{Bibliography}
		\begin{itemize}
			\item \textit{Basic MVC Architecture}. [Online]. Available: <http://www.tutorialspoint.com/struts\_2/basic\\\_mvc\_architecture.htm> [Accessed 5 March 2015].
			
			\item Beal, V. n.d. \textit{Linux Server}. [Online]. Available: <http://www.webopedia.com/TERM/L/linux\_\\server.html> [Accessed 10 March 2015].

			\item Bien, A. 2009. \textit{9 Reasons Why Java EE 6 Will Save Real Money - Or How To Convince Your Management}. [Online]. Available: <http://www.adam-bien.com/roller/abien/entry/\\8\_reasons\_why\_java\_ee> [Accessed 10 March 2015].

			\item Borges, B. 2013. \textit{Reasons to why I'm reconsidering JSF}. [Online]. Available: <http://blog.brunoborges.com.br/2013/01/reasons-to-why-im-reconsidering-jsf.html> [Accessed 10 March 2015].
			
			\item Davids, N.\textit{The Limitations of the Ethernet CRC and TCP/IP checksums for error detection} [Online]. Available: 
			<http://noahdavids.org/self-published/CRC-and-checksum.html>[Accessed 8 March 2015].

			\item \textit{Git --loval-branding-on-the-cheap}. [Online]. Available: <http://git-scm.com/> [Accessed 10 March 2015].

			\item \textit{JavaServer Faces Technology}. [Online]. Available: <http://www.oracle.com/technetwork/java/javaee/javaserverfaces-139869.html> [Accessed 10 March 2015].

			\item Kabanov, J. 2011. \textit{Ed Burns on Why JSF is the Most Popular Framework}. [Online]. Available: <http://zeroturnaround.com/rebellabs/ed-burns-on-why-jsf-is-the-most-popular-framework/> [Accessed 10 March 2015].
			\item Meyers, J. 1999. \textit{SMTP Service Extension for Authentication} [Online]. Available: <http://tools.ietf.org/html/rfc2554> [Accessed 9 March 2015]

			\item \textit{Securing FTP using SSH} [Online]. Available: <http://nurdletech.com/linux-notes/ftp/ssh.html> [Accessed 9 March 2015]

			\item \textit{PostgreSQL: About}. [Online]. Available: <http://www.postgresql.org/about/> [Accessed 10 March 2015].
			
			\item \textit{Software Engineering}. [Online]. Available: <http://sesolution.blogspot.com/p/software-engineering-layered-technology.html>[Accessed 5 March 2015].
								
			\item Solms, F. 2014. \textit{Software Architecture Desgin.} [Online]. University of Pretoria: Pretoria. Available: <http://www.cs.up.ac.za/modules/courses/download.php?id=8565> [Accessed 9 March 2015].
			
			\item Solms, F. 2015. \textit{Java Persistence API (JPA).} [Online]. University of Pretoria: Pretoria. Available: <http://www.cs.up.ac.za/modules/courses/download.php?id=8640> [Accessed 7 March 2015].
			
			\item \textit{SQL Server 2008 R2 Express Database Size Limit Increased to 10GB}. [Online]. Available: <http://blogs.msdn.com/b/sqlexpress/archive/2010/04/21/database-size-limit-increased-to-10gb-in-sql-server-2008-r2-express.aspx> [Accessed 9 March 2015].
			
			\item \textit{Usability Evaluation Basics}. [Online]. Available: <http://www.usability.gov/what-and-why/usability-evaluation.html> [Accessed 5 March 2015].

			\item W�hner, K. n.d. \textit{Why I will use Java EE (JEE, and not J2EE) instead of Spring in new Enterprise Java Projects in 2012}. [Online]. Available: <www.kai-waehner.de/blog/2011/11/21/why-i-will-use-java-ee-jee-and-not-j2ee-instead-of-spring-in-new-enterprise-java-projects-in-2012/> [Accessed 10 March 2015].		

			\item \textit{Welcome to Apache Maven}. [Online]. Available: <http://maven.apache.org/> [Accessed 10 March 2015].
			

			

			

			

			

			

		\end{itemize}
\bibliography{myrefs}{} 
\bibliographystyle{ieeetr}
\end{document}